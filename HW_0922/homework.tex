\documentclass{article}
\usepackage[utf8]{inputenc}
\usepackage{setspace}
\usepackage{amssymb}
\usepackage{amsmath}
\usepackage{systeme}
\usepackage{mathtools}

\def\R{\mathbb{R}}

\begin{document}

\section{Problem 1}

\subsection{Question a}

~

Define $U : \mathrm{span} U = \mathrm{span} (S_1 \cup S_2) \land V : \mathrm{span} V=\mathrm{span}(S_1)+\mathrm{span}(S_2) \land \mathrm{span}(S_1)=\sum_{i}m_i \land \mathrm{span}(S_2)=\sum_{j}n_j$

\subsubsection{$\mathrm{span} (S_1 \cup S_2) \subseteq \mathrm{span}(S_1)+\mathrm{span}(S_2)$}

~

Let $x\in \mathrm{span}(U)$, then $x$ can be written as :

\[x=\sum_{i}a_i m_i+\sum_{j}b_j n_j\]

$\sum_{i}a_i m_i$ spans $S_1$ and $\sum_{j}b_j n_j$ spans $S_2 \Rightarrow x \in \mathrm{span}(S_1)+\mathrm{span}(S_2)$

\subsubsection{$\mathrm{span}(S_1)+\mathrm{span}(S_2) \subseteq\mathrm{span} (S_1 \cup S_2)$}

Let $y \in \mathrm{span}(V)$, then $y$ can be written as:

\[y=\sum_{i}a_i m_i+\sum_{j}b_j n_j\]

$\sum_{i}a_i m_i$ spans $S_1$ and $\sum_{j}b_j n_j$ spans $S_2 \Rightarrow$ y is a linear combination of $S_1 \cup S_2 \Rightarrow x \in \mathrm{span}(S_1)+\mathrm{span}(S_2)$

\subsubsection{Conclusion}

~

\begin{equation}
\tag{1.1.3}
\begin{split}
&\mathrm{span} (S_1 \cup S_2) \subseteq \mathrm{span}(S_1)+\mathrm{span}(S_2)\\
&\mathrm{span}(S_1)+\mathrm{span}(S_2) \subseteq\mathrm{span} (S_1 \cup S_2)\\
\Rightarrow & \mathrm{span} (S_1 \cup S_2) = \mathrm{span}(S_1)+\mathrm{span}(S_2)
\end{split}
\end{equation}

\subsection{Question b}

~

According to the question: $W_2={A | A^{T}=-A} $.

Define $m\in W_2 \land n \in W_2\land a\in \R$

\subsubsection{Zero}

~

\begin{equation}
\tag{1.2.1}
\begin{split}
m &= \overrightarrow{0}\\
-m &= \overrightarrow{0}\\
m^T &= \overrightarrow{0}\\
\Rightarrow m^T=-m\\
\end{split}
\end{equation}


\subsubsection{Addition}

~

\begin{equation}
\tag{1.2.2}
\begin{split}
m \boxplus n &= \begin{bmatrix}
m_{11}+n{11}&...&m_{1n}+n_{1n}\\
...&...&...\\
m_{n1}+n{n1}&...&m_{nn}+n_{nn}\\
\end{bmatrix}\\
(m \boxplus n)^T&= \begin{bmatrix}
m_{11}+n{11}&...&m_{n1}+n_{n1}\\
...&...&...\\
m_{1n}+n{1n}&...&m_{nn}+n_{nn}\\
\end{bmatrix}\\
&= \begin{bmatrix}
m_{11}&...&m_{n1}\\
...&...&...\\
m_{1n}&...&m_{nn}
\end{bmatrix}
\boxplus \begin{bmatrix}
n_{11}&...&n_{n1}\\
...&...&...\\
n_{1n}&...&n_{nn}\\
\end{bmatrix}\\
&=\begin{bmatrix}
-m_{11}&...&-m_{1n}\\
...&...&...\\
-m_{n1}&...&-m_{nn}
\end{bmatrix}
\boxplus \begin{bmatrix}
-n_{11}&...&-n_{1n}\\
...&...&...\\
-n_{n1}&...&-n_{nn}\\
\end{bmatrix}\\
&=-m \boxplus -n \\
&=- (m\boxplus n)\\
\Rightarrow (m\boxplus n)^T &= -(m\boxplus n)\\
\end{split}
\end{equation}

\subsubsection{Multiplication}

~

\begin{equation}
\tag{1.2.3}
\begin{split}
(a \boxdot m)^T &= \begin{bmatrix}
am_{11}&...&am_{n1}\\
...&...&...\\
am_{1n}&...&am_{nn}
\end{bmatrix}\\
&= a \boxdot \begin{bmatrix}
m_{11}&...&m_{n1}\\
...&...&...\\
m_{1n}&...&m_{nn}
\end{bmatrix}\\
&= a \boxdot \begin{bmatrix}
-m_{11}&...&-m_{1n}\\
...&...&...\\
-m_{n1}&...&-m_{nn}
\end{bmatrix}\\
&= -a \boxdot \begin{bmatrix}
m_{11}&...&m_{1n}\\
...&...&...\\
m_{n1}&...&m_{nn}
\end{bmatrix}\\
&= -a \boxdot m\\
\Rightarrow (a \boxdot m)^T &= -a \boxdot m\\
\end{split}
\end{equation}

\subsubsection{Conclusion}

~

Since $W_2$ with these operations are valid under addition and multiplication with $\overrightarrow{0} \in W_2$, $W_2$ is a subspace of $M_{n\times n(\R)}$.

\subsection{Question c}

\subsubsection{$M_{n\times n}(\R)=W_1+W_2$}

~

Define $A \in M_{n\times n}(\R)$

$$
A = \frac{1}{2}(A+A^T) + \frac{1}{2}(A-A^T)
$$

In this place, $A$ is an arbitrary matrix.

Define $\frac{1}{2}(A+A^T)=M \land \frac{1}{2}(A-A^T)=N$

$$
A = M + N
$$

\begin{equation}
\tag{1.3.1-1}
\begin{split}
M^T &= (\frac{1}{2}(A+A^T))^T\\
&= \frac{1}{2}(A+A^T)^T\\
&= \frac{1}{2}(A^T+(A^T)^T)\\
&= \frac{1}{2}(A^T+A)\\
&= M\\
\Rightarrow &M\in W_1\\
\end{split}
\end{equation}

\begin{equation}
\tag{1.3.1-2}
\begin{split}
N^T &= (\frac{1}{2}(A-A^T))^T\\
&=\frac{1}{2}(A-A^T)^T\\
&=\frac{1}{2}(A^T-(A^T)^T)\\
&=\frac{1}{2}(A^T-A)\\
&= -\frac{1}{2}(A-A^T)\\
&= -N\\
\Rightarrow &N\in W_2\\
\end{split}
\end{equation}

\begin{equation}
\tag{1.3.1-3}
\begin{split}
A &= M+N\\
A &\in M_{n\times n}(\R)\\
M &\in W_1\\
N &\in W_2\\
\Rightarrow M_{n\times n}(\R) &= W_1 + W_2\\
\end{split}
\end{equation}

\subsubsection{$W_1\cap W_2$}

~

Define $P = W_1\cap W_2$

\begin{equation}
\tag{1.3.2}
\begin{split}
P &= \{A | A^T=A \cap A^T=-A\}\\
\Rightarrow P &= \{A|A=-A\}\\
\Rightarrow P &= \{A|A=\overrightarrow{0}\}\\
\Rightarrow P &= \overrightarrow{0}\\
\Rightarrow W_1\cap W_2 &= \overrightarrow{0}\\
\end{split}
\end{equation}

\newpage

\section{Problem 2}

\subsection{Question a}

~

Define $s=(1,0,1,1)\land t=(2,0,2,3)$

If $v \in \mathrm{span}(S)$, then $v = m\cdot s+n\cdot t$, where $m\in \R\land b\in\R$.

$$
m\cdot s+n\cdot t = (m+2n,0,m+2n,m+3n)
$$

\begin{equation}
\label{2.1}
\tag{2.1}
\begin{cases}
m+2n=0\\
0=1\\
m+2n=4\\
m+3n=2\\
\end{cases}
\end{equation}

It is easy to see that there is no solution for the set of equations \ref{2.1} $\Rightarrow v\notin \mathrm{span(S)}$

\subsection{Question b}

~

If $v\in \mathrm{span}(S)$, then $v=a(x^3)+b(2x+x^2)+c(x+x^3)$, where $a\in \R\land b\in\R\land c\in\R$

$$
a(x^3)+b(2x+x^2)+c(x+x^3)=x(2b+c)+x^2(b)+x^3(a+c)
$$

\begin{equation}
\tag{2.2}
\begin{split}
&\begin{cases}
2b+c=1\\
b=0\\
a+c=-1\\
\end{cases}\\
\Rightarrow &\begin{cases}
a=-2\\
b=0\\
c=1\\
\end{cases}\\
\Rightarrow &v=-2(x^3)+(x+x^3)\\
\Rightarrow &v\in \mathrm{span}(S)\\
\end{split}
\end{equation}

\subsection{Question c}

~

$$
v = 1+\cos(2x) = 1+ 2\cos^2(x)-1 = 2\cos^2(x)
$$

~

\begin{center}
Define $s = \sin^2(x) \land t= \cos^2(x)$
\end{center}

$$
v=2t
\Rightarrow v\in \mathrm{span}(s)
$$

\newpage

\section{Problem 3}

\subsection{Question a}

~

Define $S=a(1,2,-1)+b(2,-3,1)+c(2,3,-5)$,where $a \in \R\land b\in\R\land c\in \R$.

\begin{equation}
\tag{3.1-1}
S = \begin{bmatrix}
1&2&2\\
2&-3&3\\
-1&1&-5\\
\end{bmatrix}
\begin{bmatrix}
a\\
b\\
c\\
\end{bmatrix}
\end{equation}

\begin{equation}
\tag{3.1-2}
\begin{split}
&\begin{cases}
a+2b+2c=0\\
2a-3b+3c=0\\
-a+b-5c=0\\
\end{cases}\\
\Rightarrow &\begin{cases}
a=0\\
b=0\\
c=0\\
\end{cases}\\
\end{split}
\end{equation}

Since $a=0\land b=0\land c=0$, $S$ is linear independent.

\subsection{Question b}

~

Define $S=a(1+x)+b(1+x^2)+c(x+x^2)$,where $a \in \R\land b\in\R\land c\in \R$.

\begin{equation}
\tag{3.2-1}
S = \begin{bmatrix}
1&1&0\\
1&0&1\\
0&1&1\\
\end{bmatrix}
\begin{bmatrix}
1\\
x\\
x^2\\
\end{bmatrix}
\begin{bmatrix}
a\\
b\\
c\\
\end{bmatrix}
\end{equation}

\begin{equation}
\tag{3.2-2}
\begin{split}
&\begin{bmatrix}
1&1&0\\
1&0&1\\
0&1&1\\
\end{bmatrix}\\
\sim &\begin{bmatrix}
1&0&0\\
0&1&0\\
0&0&1\\
\end{bmatrix}\\
\Rightarrow &\begin{cases}
a=0\\
b=0\\
c=0\\
\end{cases}
\end{split}
\end{equation}

Since $a=0\land b=0\land c=0$, $S$ is linear independent.

\subsection{Question c}

~

Define $S=a\begin{bsmallmatrix}0&1\\1&1\end{bsmallmatrix}+b\begin{bsmallmatrix}-1&0\\1&1\end{bsmallmatrix}+c\begin{bsmallmatrix}1&1\\0&1\end{bsmallmatrix}+d\begin{bsmallmatrix}1&1\\1&0\end{bsmallmatrix}$, where $a\in\R\land b\in\R\land c\in\R\land d\in\R$

\begin{equation}
\tag{3.3-1}
S = \begin{bmatrix}
0&1&1&1\\
1&0&1&1\\
1&1&0&1\\
1&1&1&0\\
\end{bmatrix}
\begin{bmatrix}
a\\
b\\
c\\
d\\
\end{bmatrix}
\end{equation}

\begin{equation}
\tag{3.3-2}
\begin{split}
&\begin{cases}
b+c+d=0\\
a+c+d=0\\
a+b+d=0\\
a+b+c=0\\
\end{cases}\\
\Rightarrow &\begin{cases}
a=0\\
b=0\\
c=0\\
d=0\\
\end{cases}\\
\end{split}
\end{equation}

Since $a=0\land b=0\land c=0\land d=0$, $S$ is linear independent.

\newpage

\section{Problem 4}

\subsection{Question a}

~

\begin{equation}
\tag{4.1-1}
S_1=\begin{bmatrix}
1&0&0\\
1&1&0\\
1&1&1\\
\end{bmatrix}
\begin{bmatrix}
u\\
v\\
w\\
\end{bmatrix}
\end{equation}

\begin{equation}
\tag{4.1-2}
\begin{split}
&\begin{bmatrix}
1&0&0\\
1&1&0\\
1&1&1\\
\end{bmatrix}\\
\sim & \begin{bmatrix}
1&0&0\\
0&1&0\\
0&0&1\\
\end{bmatrix}\\
\Rightarrow & \begin{cases}
u=0\\
v=0\\
w=0\\
\end{cases}\\
\end{split}
\end{equation}

Since $u=0\land v=0\land w=0$, $S_1$ is linear independent.

\subsection{Question b}

~

\begin{equation}
\tag{4.2-1}
S_2=\begin{bmatrix}
1&-1&0\\
0&1&-1\\
-1&0&1\\
\end{bmatrix}
\begin{bmatrix}
u\\
v\\
w\\
\end{bmatrix}
\end{equation}

\begin{equation}
\tag{4.2-2}
\begin{split}
&\begin{bmatrix}
1&-1&0\\
0&1&-1\\
-1&0&1\\
\end{bmatrix}\\
\sim &\begin{bmatrix}
1&-1&0\\
0&1&-1\\
0&0&0\\
\end{bmatrix}\\
\end{split}
\end{equation}

We can see from that the set of equations have infinite solutions.
$\Rightarrow$ The vectors are linear dependent.

\newpage

\section{Problem 5}

\subsection{Question a}

~

\begin{equation}
\tag{5.1}
\begin{split}
&\begin{bmatrix}
1&0&-1&0&-3\\
0&2&2&-1&3\\
-1&0&0&1&0\\
\end{bmatrix}\\
\sim & \begin{bmatrix}
1&0&0&-1&0\\
0&1&0&\frac{1}{2}&\frac{9}{2}\\
0&0&1&-1&3\\
\end{bmatrix} \rightarrow \text{RREF}\\
\end{split}
\end{equation}

Since there are three pivot columns in the RREF, these vectors span $\R^3$.

\subsection{Question b}

Define $a\in\R\land b\in\R\land c\in\R\land d\in\R\land e\in\R$.

\begin{equation}
\tag{5.2}
\begin{split}
&\begin{bmatrix}
1&0&-1&0&-3\\
0&2&2&-1&3\\
-1&0&0&1&0\\
\end{bmatrix}
\begin{bmatrix}
a\\
b\\
c\\
d\\
e\\
\end{bmatrix}=
\begin{bmatrix}
0\\
0\\
0\\
0\\
0\\
\end{bmatrix}\\
&\begin{cases}
a-c+3e=0\\
2b+2c-d+3e=0\\
-a+d=0\\
\end{cases}\\
\Rightarrow &a=d=c-5e \land b=-\frac{c+6e}{2}\\
\end{split}
\end{equation}

The equations have infinite solutions.
$\Rightarrow$ The vectors are linear dependent.

\subsection{Question c}

~

\begin{equation}
\tag{5.3-1}\
\begin{split}
B &= \{u_1,u_3,u_5\}\\
&=\begin{bmatrix}
1&-1&3\\
0&2&3\\
-1&0&0\\
\end{bmatrix}\\
\end{split}
\end{equation}

\begin{equation}
\tag{5.3-2}
\begin{split}
&\begin{bmatrix}
1&-1&3\\
0&2&3\\
-1&0&0\\
\end{bmatrix}\\
\sim &\begin{bmatrix}
1&0&\frac{9}{2}\\
0&2&3\\
0&0&9\\
\end{bmatrix} \rightarrow \text{RREF}\\
\end{split}
\end{equation}

Since the RREF form has 3 pivot columns, the vectors form a basis of $\R^3$.

\newpage

\section{Reference}

\subsection{Collaborators}

~

Frank Zhu

~

Jeffery Shu
\end{document}