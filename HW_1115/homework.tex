\documentclass{article}
\usepackage[utf8]{inputenc}
\usepackage{setspace}
\usepackage{amssymb}
\usepackage{amsmath}
\usepackage{amsthm}
\usepackage{systeme}
\usepackage{mathtools}

\def\R{\mathbb{R}}
\def\rank{\text{rank}}
\DeclareMathOperator{\im}{im}

\begin{document}

\section{Problem 1}

\subsection{Question a}

~

\begin{equation*}
    \begin{split}
        &\begin{split}
            T(a,b)=&(-2a+b,-10a+9b)\\
            A=&\begin{bmatrix}
                -2&1\\
                -10&9\\
            \end{bmatrix}\\
            \det(A-I\lambda)=&(-2-\lambda)(9-\lambda)-(-10)\\
            =&\lambda^2-7\lambda-8\\
            =&0\\
            \Rightarrow&\lambda=-1\lor\lambda=8\\
        \end{split}\\
        &\begin{split}
            &\lambda=-1\\
            &(A-I\lambda)v=\overrightarrow{0}\\
            \Rightarrow&\begin{bmatrix}
                -1&1\\
                -10&10\\
            \end{bmatrix}v=\overrightarrow{0}\\
            \Rightarrow&v=\begin{bmatrix}
                1\\
                1\\
            \end{bmatrix}\\
            &\lambda=-8\\
            &(A-I\lambda)v=\overrightarrow{0}\\
            \Rightarrow&\begin{bmatrix}
                -10&1\\
                -10&1\\
            \end{bmatrix}v=\overrightarrow{0}\\
            \Rightarrow&v=\begin{bmatrix}
                1\\
                10\\
            \end{bmatrix}\\
            \Rightarrow&\beta=\{a+b,a+10b\}\\
        \end{split}
    \end{split}
\end{equation*}

\subsection{Question b}

\begin{equation*}
    \begin{split}
        &B={1,x,x^2,x^3}\\
        &T(1)=-1\\
        &T(x)=x-2\\
        &T(x^2)=2x^2-2\\
        &T(x^3)=3x^3+6x-8\\
        &A=\begin{bmatrix}
            -1&0&0&0\\
            -2&1&0&0\\
            -2&0&2&0\\
            -8&6&0&3\\
        \end{bmatrix}\\
        &\det(A-I\lambda)=0\\
        \Rightarrow&(-1-\lambda)(1-\lambda)(2-\lambda)(3-\lambda)=0\\
        \Rightarrow&\lambda=-1\lor\lambda=1\lor\lambda=2\lor\lambda=3\\
        &\lambda=-1\\
        &\begin{bmatrix}
            0&0&0&0\\
            -2&2&0&0\\
            -2&0&3&0\\
            -8&6&0&4\\
        \end{bmatrix}v=\overrightarrow{0}\\
        \Rightarrow&v=\begin{bmatrix}
            1\\
            1\\
            \frac{2}{3}\\
            \frac{1}{2}\\
        \end{bmatrix}\\
        &\lambda=1\\
        &\begin{bmatrix}
            -2&0&0&0\\
            -2&0&0&0\\
            -2&0&1&0\\
            -8&6&0&2\\
        \end{bmatrix}v=\overrightarrow{0}\\
        \Rightarrow&v=\begin{bmatrix}
            0\\
            -\frac{1}{3}\\
            0\\
            1\\
        \end{bmatrix}\\
        &\lambda=2\\
        &\begin{bmatrix}
            -3&0&0&0\\
            -2&-1&0&0\\
            -2&0&0&0\\
            -8&6&0&1\\
        \end{bmatrix}v=\overrightarrow{0} \\
        \Rightarrow&v=\begin{bmatrix}
            0\\
            0\\
            1\\
            0\\
        \end{bmatrix}\\
    \end{split}
\end{equation*}

\begin{equation*}
    \begin{split}
        \lambda=3\\
        &\begin{bmatrix}
            -4&0&0&0\\
            -2&-2&0&0\\
            -2&0&-1&0\\
            -8&6&0&0\\
        \end{bmatrix}v=\overrightarrow{0}\\
        \Rightarrow&v=\begin{bmatrix}
            0\\
            0\\
            0\\
            1\\
        \end{bmatrix}\\
        \Rightarrow&\beta=\{1+x+\frac{2}{3}x^2+\frac{1}{2}x^3,-\frac{1}{3}x+x^3,x^2,x^3\}
    \end{split}
\end{equation*}

\newpage

\section{Problem 2}

\subsection{Question a}

~

\begin{equation*}
    \begin{split}
        &B=\{1,x,x^2\}\\
        &T(1)=3\\
        &T(x)=3x+1\\
        &T(x^2)=4x^2\\
        \Rightarrow&A=\begin{bmatrix}
            3&0&0\\
            1&3&0\\
            0&0&4\\
        \end{bmatrix}\\
        &\det(A-I\lambda)=0\\
        \Rightarrow&(3-\lambda)^2(4-\lambda)=0\\
        \Rightarrow&\lambda=3\lor\lambda=4\\
        &\lambda=3\\
        &\begin{bmatrix}
            0&0&0\\
            1&0&0\\
            0&0&1\\
        \end{bmatrix}v=\overrightarrow{0}\\
        \Rightarrow&v=\begin{bmatrix}
            0\\
            1\\
            0\\
        \end{bmatrix}\\
        &\lambda=4\\
        &\begin{bmatrix}
            -1&0&0\\
            1&-1&0\\
            0&0&0\\
        \end{bmatrix}v=\overrightarrow{0}\\
        \Rightarrow&v=\begin{bmatrix}
            0\\
            0\\
            1\\
        \end{bmatrix}\\
        &\text{Eigenvectors cannot form a basis for }A\\
        &T\text{ is not diagonizable}\\
    \end{split}
\end{equation*}

\subsection{Question b}

~

\begin{equation*}
    \begin{split}
        &\det(A-I\lambda)=0\\
        &(-1-\lambda)\det\begin{bmatrix}
            -2-\lambda&-3&1\\
            1&2-\lambda&-1\\
            -3&-3&2-\lambda\\
        \end{bmatrix}=0\\
        \Rightarrow&(\lambda+1)^2(\lambda-1)(\lambda+2)=0\\
        \Rightarrow&\lambda=-1\lor\lambda=1\lor\lambda=-2\\
        &\lambda=-1\\
        &\begin{bmatrix}
            0&-4&-4&3\\
            0&-1&-3&1\\
            0&1&-1&-1\\
            0&-3&-3&3\\
        \end{bmatrix}v=\overrightarrow{0}\\
        \Rightarrow&v=\begin{bmatrix}
            1\\
            0\\
            0\\
            0\\
        \end{bmatrix}\\
        &\lambda=1\\
        &\begin{bmatrix}
            -2&-4&-4&3\\
            0&-3&-3&1\\
            0&1&-3&-1\\
            0&-3&-3&1\\
        \end{bmatrix}v=\overrightarrow{0}\\
        \Rightarrow&v=\begin{bmatrix}
            5\\
            3\\
            -1\\
            6\\
        \end{bmatrix}\\
        &\lambda=-2\\
        &\begin{bmatrix}
            1&-4&-4&3\\
            0&0&-3&1\\
            0&1&0&-1\\
            0&-3&-3&4\\
        \end{bmatrix}v=\overrightarrow{0}\\
        \Rightarrow&v=\begin{bmatrix}
            7\\
            3\\
            1\\
            3\\
        \end{bmatrix}\\
        &\text{Eigenvectors cannot form a basis for }A\\
        &T\text{ is not diagonizable}\\
    \end{split}
\end{equation*}

\newpage

\section{Problem 3}

\subsection{Question a}

~

\begin{equation*}
    \begin{split}
        &A=SDS^{-1}\\
        \Leftrightarrow &AS=SD\\
        \Leftrightarrow &S=A^{-1}SD\\
        \Leftrightarrow &SD^{-1}=A^{-1}S\\
        \Leftrightarrow &A^{-1}=SD^{-1}S^{-1}\\
        \therefore &A=SDS^{-1}\Leftrightarrow A^{-1}=SD^{-1}S^{-1}\\
        \Rightarrow &A\text{ is diagonizable } \Leftrightarrow \text{ } A^{-1} \text{ is diagonizable}
    \end{split}
\end{equation*}

\subsection{Question b}

~

\begin{equation*}
    \begin{split}
        &A=SDS^{-1}\\
        &A^2=SD^2S^{-1}\\
        &A^3=SD^3S^{-1}\\
        &A^3+A^2+A^1+I\\
        =&SD^3S^{-1}+SD^2S^{-1}+SDS^{-1}+SIS^{-1}\\
        =&S(D^3S^{-1}+D^2S^{-1}+DS^{-1}+IS^{-1})\\
        =&S(D^3+D^2+D+I)S^{-1}\\
        \therefore& A^3+A^2+A^1+I\text{ is diagonizable}\\
    \end{split}
\end{equation*}

\subsection{Question c}

~

\begin{equation*}
    \begin{split}
        &A=SDS^{-1}\\
        \Leftrightarrow &A^T=(SDS^{-1})^T\\
        \Leftrightarrow &A^T=(S^{-1})^TD^TS^T\\
        &P\coloneqq(S^{-1})^T\\
        &PP^{-1}=I\\
        &(S^{-1})^TP^{-1}=I\\
        &((S^{-1})^TP^{-1})^T=I\\
        &(P^{-1})^TS^{-1}=I\\
        &(P^{-1})^T=S\\
        &P^{-1}=S^T\\
        \Rightarrow&A^T=PD^TP^{-1}\\
        \therefore &A=SDS^{-1}\Leftrightarrow A^T=PD^TP^{-1}\\
        &A\text{ is diagonizable } \Leftrightarrow \text{ } A^T \text{ is diagonizable}
    \end{split}
\end{equation*}

\newpage

\section{Problem 4}

\subsection{Question a}

~

\begin{equation*}
    \begin{split}
        Tv=&\lambda v\\
        Tv=&T^2v\\
        =&T\cdot Tv\\
        =&T\lambda v\\
        =&\lambda Tv\\
        =&\lambda \lambda v\\
        =&\lambda^2v\\
        =&\lambda v\\
        \Rightarrow &\lambda^2-\lambda=0 \rightarrow v\ne 0\\
        \Rightarrow &\lambda = 0\lor \lambda=1\\
        &\text{Since all the eigenvalues are distinct, }T\text{ is diagonizable}\\
    \end{split}
\end{equation*}

\subsection{Question b}

~

\begin{equation*}
    \begin{split}
        &T(V)\coloneqq WV\\
        \Rightarrow &WV=V^T\\
        &T(WV)=(V^T)^T=V=WWV\\
        \Rightarrow &WW=I\\
        &Wv=\lambda v\\
        \Rightarrow &WWv=W\lambda v=\lambda Wv=\lambda^2v\\
        &WW=I\Rightarrow WWv=v\\
        \Rightarrow \lambda^2v=v\\
        \Rightarrow \lambda =\pm 1\\
        &\text{Since all the eigenvalues are distinct, }T\text{ is diagonizable}\\
    \end{split}
\end{equation*}

\newpage

\section{Problem 5}

~

\subsection{Question a}

~

\begin{equation*}
    \begin{split}
        &D(f(x))\coloneqq Af(x)\\
        &D(e^{\lambda x})=Ae^{\lambda x}\\
        &(e^{\lambda x})'=\lambda e^{\lambda x}\\
        \Rightarrow &Ae^{\lambda x}=\lambda e^{\lambda x}\\
        &Af_\lambda(x)=\lambda f_\lambda(x)\\
        \Rightarrow& f_\lambda(x)\text{ are eigenvectors}\\
    \end{split}
\end{equation*}

\subsection{Question b}

~

\begin{equation*}
    \begin{split}
        &e^{\lambda x} = \sum^\infty_{n=0}\frac{x^n}{n!}\\
        &\text{Take the first }n-1 \text{ terms of the expansion}{T^{n-1}}_\lambda\text{ as the approximation to } e^{\lambda x}\\
        &\begin{bmatrix}
            f_{\lambda_{1}}x\\
            f_{\lambda_{2}}x\\
            f_{\lambda_{3}}x\\
            \vdots\\
            f_{\lambda_{n}}x\\
        \end{bmatrix}\approx \begin{bmatrix}
            {T^{n-1}}_{\lambda_1}\\
            {T^{n-1}}_{\lambda_2}\\
            {T^{n-1}}_{\lambda_3}\\
            \vdots\\
            {T^{n-1}}_{\lambda_n}\\
        \end{bmatrix}=\underbrace{\begin{bmatrix}
            1&\lambda_{1}x&(\lambda_{1}x)^2&\cdots&(\lambda_{1}x)^{n-1}\\
            1&\lambda_{2}x&(\lambda_{2}x)^2&\cdots&(\lambda_{2}x)^{n-1}\\
            1&\lambda_{3}x&(\lambda_{3}x)^2&\cdots&(\lambda_{3}x)^{n-1}\\
            \vdots &\vdots&\vdots&\ddots &\vdots \\
            1&\lambda_{n}x&(\lambda_{n}x)^2&\cdots&(\lambda_{n}x)^{n-1}\\
        \end{bmatrix}}_V\begin{bmatrix}
            \frac{1}{0!}\\
            \frac{1}{1!}\\
            \frac{1}{2!}\\
            \vdots\\
            \frac{1}{(n-1)!}\\
        \end{bmatrix}\\
        &\det(V)=\prod_{0\leqslant i<j\leqslant n}(\lambda_jx-\lambda_ix)\\
        &\lambda_1,\ldots ,\lambda_n\text{ are distinct values}\implies \lambda_jx-\lambda_ix\ne0\\
        \Rightarrow&\det(V)\ne0\\
        \Rightarrow &V \text{ is linear independent}\\
        &\text{By induction: }\nexists k\in\mathbb{Z}^+ : \exists m\in\mathbb{Z}^+: {T^k}_{\lambda_m}=\sum_{i=0}^{m-1}a_i{T^k}  _{\lambda_{m-1}}\\
        \Rightarrow &\begin{bmatrix}
            f_{\lambda_{1}}x\\
            f_{\lambda_{2}}x\\
            \vdots\\
            f_{\lambda_{n}}x\\
        \end{bmatrix}\text{ is linear independent}\\
    \end{split}
\end{equation*}

\subsection{Question c}

~

\begin{equation*}
    \begin{split}
        &\text{Set }D(f(x))=Af(x)=\lambda f(x)\\
        &D(f(x))=f'(x)\\
        \Rightarrow&f'(x)=\lambda f(x)\\
        &y\coloneqq f(x)\\
        \Rightarrow&\frac{dy}{dx} -\lambda y=0\\
        &I=e^{\int -\lambda \,dx }\\
        \Rightarrow&I=e^{-\lambda x}\\
        &e^{-\lambda x}\frac{dy}{dx}-e^{-\lambda x}\lambda y=0\\
        &\frac{d}{dx}(e^{-\lambda x}y)=0\\
        &e^{-\lambda x}y=C\\
        &y=Ce^{\lambda x}\\
        \Rightarrow& f(x)=Cf_\lambda (x)\\
    \end{split}
\end{equation*}

\subsection{Question 4}

~

\begin{equation*}
    \begin{split}
        &\forall \lambda \in \mathbb{Z}^+ \land C \in \mathbb{R} : Af(x)=\lambda f(x), \text{ where }f(x)=Ce^{\lambda x}\\
        &\text{But }Ce^{\lambda x}\text{ are linear dependent, with only one independent vector}\\
        \Rightarrow&\text{There is one eigenvector }f_\lambda(x)=e^{\lambda x}\text{ for every }\lambda\\
        &\text{As proved before: }f_\lambda(x)\text{ with different }\lambda\text{ are linear independent}\\
        \Rightarrow&\text{It can form an ordered basis for }D\\
        \Rightarrow&D\text{ is diagonizable}
    \end{split}
\end{equation*}

\newpage

\section{Reference}

~

Jeffery Shu

~

Frank Zhu
\end{document}