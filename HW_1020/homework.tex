\documentclass{article}
\usepackage[utf8]{inputenc}
\usepackage{setspace}
\usepackage{amssymb}
\usepackage{amsmath}
\usepackage{amsthm}

\def\R{\mathbb{R}}
\def\Trace{\text{Trace}}

\begin{document}

\section{Problem 1}

~

\begin{equation*}
\begin{split}
T(p_1)=&(p_1(1),\int^1_0p_1(x)dx)\\
=&(3,\left[\frac{1}{3}x^3+\frac{1}{2}x^2+x\right]^1_0)\\
=&(3,\frac{11}{6})\\
=&-\frac{11}{6}(1,-1)+\frac{29}{12}(2,0)\\
T(p_2)=&(p_2(1),\int^1_0p_2(x)dx)\\
=&(2,\left[\frac{1}{2}x^2+x\right]^1_0)\\
=&(2,\frac{3}{2})\\
=&-\frac{3}{2}(1,-1)+\frac{7}{4}(2,0)\\
T(p_3)=&(p_3(1),\int^1_0p_3(x)dx)\\
=&(1,\left[x\right]^1_0)\\
=&(1,1)\\
=&-(1,-1)+(2,0)\\
\Rightarrow \left[T\right]^{\gamma_1}_{\beta_1}=&\begin{bmatrix}
-\frac{11}{6}&-\frac{3}{2}&-1\\
\frac{29}{12}&\frac{7}{4}&1\\
\end{bmatrix}\\
\end{split}
\end{equation*}

\begin{equation*}
\begin{split}
T(f_1)=&(f_1(1),\int^1_0f_1(x)dx)\\
=&(0,\left[\frac{1}{3}x^3-\frac{1}{2}x^2\right]^1_0)\\
=&(0,-\frac{1}{6})\\
=&\frac{1}{6}(-1,2)+\frac{1}{6}(1,-3)\\
T(f_2)=&(f_2(1),\int^1_0f_2(x)dx)\\
=&(2,\left[\frac{1}{2}x^2+x\right]^1_0)\\
=&(2,\frac{3}{2})\\
=&-\frac{15}{2}(-1,2)-\frac{11}{2}(1,-3)\\
T(f_3)=&(f_3(1),\int^1_0f_3(x)dx)\\
=&(2,\left[x^2\right]^1_0)\\
=&(2,1)\\
=&-7(-1,2)-5(1,-3)\\
\Rightarrow \left[T\right]^{\gamma_2}_{\beta_2}=&\begin{bmatrix}
\frac{1}{6}&-\frac{15}{2}&-7\\
\frac{1}{6}&-\frac{11}{2}&-5\\
\end{bmatrix}\\
\end{split}
\end{equation*}

\begin{equation*}
\begin{split}
p_1(x)=&x^2+x+1\\
=&1\cdot(x^2-x)+1\cdot(x+1)+\frac{1}{2}(2x)\\
p_2(x)=&x+1\\
=&0\cdot(x^2-x)+1\cdot(x+1)+0\cdot(2x)\\
p_3(x)=&1\\
=&0\cdot(x^2-x)+1\cdot(x+1)-\frac{1}{2}\cdot(2x)\\
\Rightarrow Q^{\beta_2}_{\beta_1}=&\begin{bmatrix}
1&0&0\\
1&1&1\\
\frac{1}{2}&0&-\frac{1}{2}\\
\end{bmatrix}\\
\end{split}
\end{equation*}

\begin{equation*}
\begin{split}
(-1,2)=&-2\cdot(1,-1)+\frac{1}{2}\cdot(2,0)\\
(1,-3)=&3\cdot(1,-1)-1\cdot(2,0)\\
\Rightarrow Q^{\gamma_1}_{\gamma_2}=&\begin{bmatrix}
-2&3\\
\frac{1}{2}&-1\\
\end{bmatrix}\\
\end{split}
\end{equation*}

\begin{equation*}
\begin{split}
&Q^{\gamma_1}_{\gamma_2} \left[T\right]^{\gamma_2}_{\beta_2} Q^{\beta_2}_{\beta_1}\\
=&\begin{bmatrix}
-2&3\\
\frac{1}{2}&-1\\
\end{bmatrix}
\begin{bmatrix}
\frac{1}{6}&-\frac{15}{2}&-7\\
\frac{1}{6}&-\frac{11}{2}&-5\\
\end{bmatrix}
\begin{bmatrix}
1&0&0\\
1&1&1\\
\frac{1}{2}&0&-\frac{1}{2}\\
\end{bmatrix}\\
=&\begin{bmatrix}
\frac{1}{6}&-\frac{3}{2}&-1\\
\frac{1}{12}&\frac{27}{4}&\frac{3}{2}\\
\end{bmatrix}
\begin{bmatrix}
1&0&0\\
1&1&1\\
\frac{1}{2}&0&-\frac{1}{2}\\
\end{bmatrix}\\
=&\begin{bmatrix}
-\frac{11}{6}&-\frac{3}{2}&-1\\
\frac{29}{12}&\frac{7}{4}&1\\
\end{bmatrix}\\
=&\left[T\right]^{\gamma_1}_{\beta_1}\\
\Rightarrow& \left[T\right]^{\gamma_1}_{\beta_1} =Q^{\gamma_1}_{\gamma_2} \left[T\right]^{\gamma_2}_{\beta_2} Q^{\beta_2}_{\beta_1}\\
\end{split}
\end{equation*}

\newpage

\section{Problem 2}

\subsection{Question a}

~

\begin{equation*}
\begin{split}
L_A(p_1)=&\begin{bmatrix}
1&1&-1\\
2&0&1\\
1&1&0\\
\end{bmatrix}\begin{bmatrix}
1\\
1\\
0\\
\end{bmatrix}\\
=&\begin{bmatrix}
2\\
2\\
2\\
\end{bmatrix}\\
=&\frac{12}{5}\begin{bmatrix}
1\\
1\\
0\\
\end{bmatrix}+\frac{2}{5}\begin{bmatrix}
1\\
-1\\
1\\
\end{bmatrix}-\frac{4}{5}\begin{bmatrix}
1\\
0\\
-2\\
\end{bmatrix}\\
L_A(p_2)=&\begin{bmatrix}
1&1&-1\\
2&0&1\\
1&1&0\\
\end{bmatrix}\begin{bmatrix}
1\\
-1\\
1\\
\end{bmatrix}\\
=&\begin{bmatrix}
-1\\
3\\
0\\
\end{bmatrix}\\
=&\frac{7}{5}\begin{bmatrix}
1\\
1\\
0\\
\end{bmatrix}-\frac{8}{5}\begin{bmatrix}
1\\
-1\\
1\\
\end{bmatrix}-\frac{4}{5}\begin{bmatrix}
1\\
0\\
-2\\
\end{bmatrix}\\
L_A(p_3)=&\begin{bmatrix}
1&1&-1\\
2&0&1\\
1&1&0\\
\end{bmatrix}\begin{bmatrix}
1\\
0\\
-2\\
\end{bmatrix}\\
=&\begin{bmatrix}
3\\
0\\
1\\
\end{bmatrix}\\
=&\frac{7}{5}\begin{bmatrix}
1\\
1\\
0\\
\end{bmatrix}+\frac{7}{5}\begin{bmatrix}
1\\
-1\\
1\\
\end{bmatrix}+\frac{1}{5}\begin{bmatrix}
1\\
0\\
-2\\
\end{bmatrix}\\
\Rightarrow& \left[L_A\right]^\beta_\beta=\begin{bmatrix}
\frac{12}{5}&\frac{7}{5}&\frac{7}{5}\\
\frac{2}{5}&-\frac{8}{5}&\frac{7}{5}\\
-\frac{4}{5}&-\frac{4}{5}&\frac{1}{5}\\
\end{bmatrix}\\
\end{split}
\end{equation*}

\subsection{Question b}

\begin{equation*}
\begin{split}
\left[L_A\right]_\beta=&\left[I\right]^\beta_\alpha\left[L_A\right]_\alpha\left[I\right]^\alpha_\beta\rightarrow \alpha \text{ is the standard basis on } \mathbb{F}^3\\
=&Q^{-1}AQ\\
\Rightarrow Q=&\left[I\right]^\alpha_\beta \\
=&\begin{bmatrix}
1&1&1\\
1&-1&0\\
0&1&-2\\
\end{bmatrix}\\
\end{split}
\end{equation*}

\newpage

\section{Problem 3}

\subsection{Question a}

~

The reflection is on the $xy$ plane, so the $x$ and $y$ coordinates are the same, with $z$ flips the sign.

So the formula is $R(x,y,z)=(x,y,-z)$

\subsection{Question b}

~

\begin{equation*}
\begin{split}
x+2y+3z=&0\\
x=&-2y-3z\\
\Rightarrow (x,y,z)=&(-2y-3z,y,z)\\
=&y(-2,1,0)+z(-3,0,1)\\
\end{split}
\end{equation*}

So the basis for $P$ is $\left\{(-2,1,0),(-3,0,1)\right\}$, the normal vector is $(1,2,3)$.

\subsection{Question c}

~

\begin{equation*}
\begin{split}
D_p=&\frac{(x,y,z)\cdot(1,2,3)}{\lvert<1,2,3>\rvert^2}\\
R_p(x,y,z)=&(x,y,z)-2D_p\cdot(1,2,3)\\
=&(x,y,z)-2\cdot\frac{(x,y,z)\cdot(1,2,3)}{\lvert<1,2,3>\rvert^2}\cdot(1,2,3)\\
\end{split}
\end{equation*}

\newpage

\section{Problem 4}

\subsection{Question a}

~

\begin{equation*}
\begin{split}
f(A+B)=&\Trace(A+B)\\
=&\Trace(A)+\Trace(B)\\
=&f(A)+f(B)\\
f(cA)=&\Trace(cA)\\
=&c\Trace(A)\\
=&cf(A)\\
\end{split}
\end{equation*}

Therefore, $\Trace(A)$ fits the properties of linear functional and in $V^*$.

\subsection{Question b}

~

\begin{proof}

Assume there exists matrices: $AB-BA=I$

\begin{equation*}
\begin{split}
\Trace(AB-BA)=&\Trace(I)\\
\Trace(AB)-\Trace(BA)=&\Trace(I)\\
\end{split}
\end{equation*}

$\Trace(AB)-\Trace(BA)$ is the trace of commutator, which is 0.

However, in this case, we assume that $\Trace(AB)-\Trace(BA)=\Trace(I)$, which is not 0.

There is the contradiction.

~

Therefore there cannot be matrices that $AB-BA=I$.
\end{proof}

\subsection{Question c}

~

Define $p(x)=g(x)-g(I)f(x)$

We need to first prove the commutative property for $p(x)$

\begin{equation*}
\begin{split}
p(AB)=&g(AB)-g(I)f(AB)\\
=&g(BA)-g(I)f(BA)\\
=&h(BA)\\
\end{split}
\end{equation*}

Then substitute $x$ as $kI$, where $k$ is an arbitrary real number and $I$ is identity matrix.

\begin{equation*}
\begin{split}
p(kI)=&g(kI)-g(I)f(kI)\\
=&kg(I)-kg(I)f(I)\\
=&0\\
\end{split}
\end{equation*}

Therefore, $p(x)=0 \forall x\in V$

\begin{equation*}
\begin{split}
g(x)-g(I)f(x)=&0\\
g(x)=&g(I)f(x)\\
\end{split}
\end{equation*}

Since $g(I)$ is a constant, $g(x)=cf(x)$
\newpage

\section{Problem 5}

\subsection{Question a}

~

\begin{equation*}
\begin{split}
f_1(x)&=\begin{cases}
1,x=-1\\
0,x\ne -1\\
\end{cases}\\
f_2(x)&=\begin{cases}
1,x=2\\
0,x\ne-2\\
\end{cases}\\
\end{split}
\end{equation*}

In this case, both functions satisfy the requirements, however, they are linear independent because they have different non-zero points, for example, there is never a $k$ for $kf_1(2)=f_2(2)$.

\subsection{Question b}

~

Assume $q_1(x)=1$ and $q_2(x)=\frac{1}{2}x-\frac{1}{4}$

\begin{equation*}
\begin{split}
f(q_1(x))&=\int^1_01dx=1\\
g(q_2(x))&=q_2(1)-q_2(-1)=1\\
f(q_2(x))&=\int^1_0\frac{1}{2}x-\frac{1}{4}dx=0\\
g(q_1(x))&=q_1(1)-q_1(-1)=0\\
\end{split}
\end{equation*}

The functions satisfy the requirements. and they are linear independent because they are in different dimensions so that they are not scalar multiples.

\newpage

\section{Reference}

\subsection{Collaborators}

~

Frank Zhu

~

Jeffery Shu
\end{document}