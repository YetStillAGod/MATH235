\documentclass{article}
\usepackage[utf8]{inputenc}
\usepackage{setspace}
\usepackage{amssymb}
\usepackage{amsmath}
\usepackage{amsthm}
\usepackage{systeme}
\usepackage{mathtools}

\def\R{\mathbb{R}}

\begin{document}

\section{Problem 1}

\subsection{Question a}

~

Define $f(x)=ax^2+bx+c \in W_1\land a,b,c\in \R$

\begin{equation}
\tag{1.1}
\begin{split}
&\int_{-1}^{1}f(x)dx=0\\
\Rightarrow &\int_{-1}^{1}ax^2+bx+c=0\\
\Rightarrow &\left[\frac{1}{3}ax^3+\frac{1}{2}bx^2+cx\right]_{-1}^{1}=0\\
\Rightarrow &(\frac{1}{3}a+\frac{1}{2}b+c)-(-\frac{1}{3}a+\frac{1}{2}b-c)=0\\
\Rightarrow &\frac{2}{3}a+2c=0\\
\Rightarrow &a=-3c\\
\Rightarrow &\text{span}(W_1)=\{\langle 3,0,1\rangle,\langle0,1,0\rangle\}\\
\Rightarrow &\text{dim}(W_1)=2
\end{split}
\end{equation}

\subsection{Question b}

~

Define $f(x)=ax^2+bx+c \in W_2\land a,b,c\in \R$

\begin{equation}
\tag{1.2}
\begin{split}
&f'(x)=2ax+b\\
&f'(1)=0\\
\Rightarrow &2a+b=0\\
\Rightarrow &b=-2a\\
\Rightarrow &\text{span}(W_2)=\{\langle1,-2,0\rangle,\langle0,0,1\rangle\}\\
\Rightarrow &\text{dim}(W_2)=2\\
\end{split}
\end{equation}

\subsection{Question c}

~

Define $f(x)=ax^2+bx+c \in W_1\cap W_2\land a,b,c\in \R$

\begin{equation}
\tag{1.3}
\begin{split}
&\text{From (1.1) and (1.2):}\\
&a=-3c \land b=-2a\\
\Rightarrow & a=-3c, b=6c,c=c\\
\Rightarrow & \text{span}(W_1\cap W_2)=\{\langle-3,6,1\rangle\}\\
\Rightarrow & \text{dim}(W_1\cap W_2) =1\\
\end{split}
\end{equation}

\newpage

\section{Problem 2}

~

\begin{proof}

Define $\mathrm{span}(W_1 \cap W_2) = \{u_1,u_2,...,u_p\}$

Since $W_1\cap W_2 \subset W_1$, we can expand the basis of $W_1\cap W_2$ to get the basis of $\mathrm{span}(W_1) =\{u_1,u_2,...,u_p,v_1,v_2,...,v_m\}$.

For the same reason, $\mathrm{span}(W_1) =\{u_1,u_2,...,u_p,w_1,w_2,...,w_n\}$

$\Rightarrow \mathrm{span}(W_1+W_2)=\{u_1,u_2,...,u_p,v_1,v_2,...,v_m,w_1,w_2,...,w_n\}$

\subsection{Linear Independence}

~

We have to show that $\{u_1,u_2,...,u_p,v_1,v_2,...,v_m,w_1,w_2,...,w_n\}$ is linear independent.

\begin{equation}
\tag{2.1}
\begin{split}
&a_1 u_1+a_2 u_2+...+a_p u_p+b_1 v_1+b_2 v_2+...+ b_m v_m+c_1 w_1+c_2 w_2+...+c_n w_n=0\\
&a_1,...,a_p,b_1,...,b_m,c_1,...,c_n\in\R\\
&\Rightarrow \sum_{i=1}^{p}a_i u_i+\sum_{j=1}^{m}b_j v_j+\sum_{k=1}^{n}c_k w_k=0\\
&\Rightarrow -\sum_{k=1}^{n}c_k w_k = \sum_{i=1}^{p}a_i u_i+\sum_{j=1}^{m}b_j v_j\\
&\Rightarrow -\sum_{k=1}^{n}c_k w_k \in W_1\\
&\text{But }-\sum_{k=1}^{n}c_k w_k\in W_2\\
&\Rightarrow -\sum_{k=1}^{n}c_k w_k\in W_1\cap W_2\\
&\Rightarrow -\sum_{k=1}^{n}c_k w_k = d_1 u_1 + d_2 u_2 +...+d_q u_q\\
& d_1,d_2,...,d_q\in \R\\
&\Rightarrow c_1 w_1+c_2 w_2+...+c_n w_n+d_1 u_1 + d_2 u_2 +...+d_q u_q=0\\
&u_1,u_2,...,u_q,w_1,w_2,...,w_n \text{ is linear independent since they span } W_2\\
&\Rightarrow d_1=d_2=...=d_q=c_1=c_2+...=c_n=0\\
&\text{For the same reason, } a_1=a_2=...=a_p=b_1=b_2=b_m=0\\
&\Rightarrow \{u_1,u_2,...,u_p,v_1,v_2,...,v_m,w_1,w_2,...,w_n\} \text{ is linear independent}\\
\end{split}
\end{equation}

\subsection{Dimension}

~

\begin{equation}
\tag{2.2-1}
\begin{split}
&\{u_1,u_2,...,u_p,v_1,v_2,...,v_m,w_1,w_2,...,w_n\} \text{ is linear independent}\\
&\Rightarrow \text{dim}(W_1+W_2)=p+m+n\\
&\text{For the same reason:}\\
&\text{dim}(W_1)=p+m\\
&\text{dim}(W_2)=p+n\\
&\text{dim}(W_1\cap W_2)=p\\
\end{split}
\end{equation}

\begin{equation}
\tag{2.2-2}
\begin{split}
\text{dim}(W_1+W_2)&=p+m+n\\
&=(p+m)+(p+n)-p\\
&=\text{dim}(W_1) + \text{dim}(W_2) -\text{dim}(W_1\cap W_2)\\
\end{split}
\end{equation}
\end{proof}

\newpage

\section{Problem 3}

~

\subsection{Question a}

~

Define $u=(a,b,c)\land v=(x,y,z)\land m\in\R$

\subsubsection{Addition}

~

\begin{equation}
\tag{3.1.1}
\begin{split}
T(u+v)&=T(a+x,b+y,c+z)\\
&=(2(a+x)+(b+y)-(c+z),(a+x)(b+y)-(c+z))\\
&=(2a+b-c+2x+y-z,ab+xy+ax+by-c-z)\\
T(u)+T(v)&=T(a,b,c)+T(x,y,z)\\
&=(2a+b-c,ab-c)+(2x+y-z,xy-z)\\
&=(2a+b-c+2x+y-z,ab+xy-c-z)\\
&\ne T(u+v)\\
\end{split}
\end{equation}

\subsubsection{Conclusion}

~

Since $T(u+v)\ne T(u)+T(v)$, $T$ is not a linear transformation.

\subsection{Question b}

~

Define $f(x)=ax^2+bx+c\land g(x)=dx^2+ex+f\land m\in\R$

\subsubsection{Addition}

~

\begin{equation}
\tag{3.2.1}
\begin{split}
T(f(x)+g(x))=&T((a+d)x^2+(b+e)x+(c+f))\\
=&x((a+d)x^2+(b+e)x+(c+f))\\
&+\int_{0}^{x}(a+d)t^2+(b+e)t+(c+f)dt\\
=&(a+d)x^3+(b+e)x^2+(c+f)x\\
&+\left[\frac{1}{3}(a+d)t^3+\frac{1}{2}(b+e)t^2+(c+f)t\right]_{0}^{x}\\
=&(a+d)x^3+(b+e)x^2+(c+f)x\\
&+\frac{1}{3}(a+d)x^3+\frac{1}{2}(b+e)x^2+(c+f)x\\
=&\frac{4}{3}(a+d)x^3+\frac{3}{2}(b+e)x^2+2(c+f)x\\
T(f(x))+T(g(x))=&x(ax^2+bx+c)+\int_{0}^{x}at^2+bt+cdt\\
&+x(dx^2+ex+f)+\int_{0}^{x}dt^2+et+fdt\\
=&ax^3+bx^2+cx+\left[\frac{1}{3}at^3+\frac{1}{2}bt^2+ct\right]_{0}^{x}\\
&+dx^3+ex^2+fx+\left[\frac{1}{3}dt^3+\frac{1}{2}et^2+ct\right]_{0}^{x}\\
=&(a+d)x^3+(b+e)x^2+(c+f)x\\
&+\frac{1}{3}(a+d)x^3+\frac{1}{2}(b+e)x^2+(c+f)x\\
=&\frac{4}{3}(a+d)x^3+\frac{3}{2}(b+e)x^2+2(c+f)x\\
=&T(f(x)+g(x))\\
\end{split}
\end{equation}

\subsubsection{Multiplication}

~

\begin{equation}
\tag{3.2.2}
\begin{split}
mT(f(x))=&m(x(ax^2+bx+c)+\int_{0}^{x}at^2+bt+cdt)\\
=&m(ax^3+bx^2+cx+\left[\frac{1}{3}at^3+\frac{1}{2}bt^2+ct\right]_{0}^{x})\\
=&m(ax^3+bx^2+cx+\frac{1}{3}ax^3+\frac{1}{2}bx^2+cx)\\
=&\frac{4}{3}amx^3+\frac{3}{2}bmx^2+2cmx\\
T(mf(x))=&x(m(ax^2+bx+c))+\int_{0}^{x}m(at^2+bt+c)dt\\
=&m(ax^3+bx^2+cx)+\left[\frac{1}{3}amt^3+\frac{1}{2}bmt^2+cmt\right]_{0}^{x}\\
=&amx^3+bmx^2+cmx+\frac{1}{3}amx^3+\frac{1}{2}bmx^2+cmx\\
=&\frac{4}{3}amx^3+\frac{3}{2}bmx^2+2cmx\\
=&mT(f(x))
\end{split}
\end{equation}

\subsubsection{Conclusion}

~

Since $T$ meets the requirement of both addition and multiplication, $T$ is a linear transformation.

\subsection{Question c}

~

Define $P,Q\in M_{n\times n}(\R)\land m\in\R$

\subsubsection{Addition}

~

\begin{equation}
\tag{3.3.1}
\begin{split}
T(P+Q)&=(P+Q)M-M(P+Q)\\
&=PM+QM-MP-MQ\\
&=PM-MP+QM-MQ\\
T(P)+T(Q)&=PM-MP+QM-MQ\\
&=T(P+Q)\\
\end{split}
\end{equation}

\subsubsection{Multiplication}

~

\begin{equation}
\tag{3.3.2}
\begin{split}
mT(P)&=m(PM-MP)\\
&=mPM-mMP\\
T(mP)&=(mP)M-M(mP)\\
&=mPM-mMP\\
&=mT(P)\\
\end{split}
\end{equation}

\subsubsection{Conclusion}

~

Since $T$ meets the requirement of both addition and multiplication, $T$ is a linear transformation.

\newpage

\section{Problem 4}

~

\subsection{Question a}

~

\begin{equation}
\tag{4.1}
\begin{split}
&T(a,b,c)\\
=&(a-2b+3c)x+(2c-a)\\
=&\begin{bmatrix}
a-2b+3c,2c-a\\
\end{bmatrix}\cdot\begin{bmatrix}
x\\
1\\
\end{bmatrix}\\
\end{split}
\end{equation}

\subsubsection{Kernel}

~

\begin{equation}
\tag{4.1.2}
\begin{split}
&\begin{cases}
a-2b+3c=0\\
2c-a=0\\
\end{cases}\\
\Rightarrow &\begin{cases}
a=2t\\
b=\frac{5}{2}t\\
c=t\\
\end{cases}\\
\Rightarrow &\text{Kernel }=\{\begin{bsmallmatrix}2\\\frac{5}{2}\\1\end{bsmallmatrix}\}\\
&\text{dim}(\text{Kernel})=1\\
\end{split}
\end{equation}

\subsubsection{Range}

~

\begin{equation}
\tag{4.1.3}
\begin{split}
&\begin{bmatrix}
a-2b+3c\\
2c-a\\
\end{bmatrix}\\
= &\begin{bmatrix}
1&-2&3\\
-1&0&2\\
\end{bmatrix}\cdot\begin{bmatrix}
a\\
b\\
c\\
\end{bmatrix}\\
&\text{Column space}\\
&\begin{bmatrix}
1&-1\\
0&-2\\
2&-3\\
\end{bmatrix}\\
\sim &\begin{bmatrix}
1&0\\
0&1\\
0&0\\
\end{bmatrix}\\
\Rightarrow &\text{Range }=\{\begin{bsmallmatrix}1\\0\end{bsmallmatrix},\begin{bsmallmatrix}0\\1\end{bsmallmatrix}\}\\
&\text{dim}(\text{Range})=2\\
\end{split}
\end{equation}

\subsection{Question b}

~

Define $A=\begin{bsmallmatrix}a&b&c\\d&e&f\\g&h&i\end{bsmallmatrix}$

\subsubsection{Kernel}

~

\begin{equation}
\tag{4.2.1}
\begin{split}
&A-A^T=0\\
\Rightarrow &A=A^T\\
\Rightarrow &A=\begin{bsmallmatrix}a&b&c\\b&e&f\\c&f&i\end{bsmallmatrix}\\
&\text{null space for symmetric matrix}:\\
& \{\begin{bsmallmatrix}1&0&0\\0&0&0\\0&0&0\end{bsmallmatrix},\begin{bsmallmatrix}0&1&0\\1&0&0\\0&0&0\end{bsmallmatrix},\begin{bsmallmatrix}0&0&1\\0&0&0\\1&0&0\end{bsmallmatrix},\begin{bsmallmatrix}0&0&0\\0&1&0\\0&0&0\end{bsmallmatrix},\begin{bsmallmatrix}0&0&0\\0&0&1\\0&1&0\end{bsmallmatrix},\begin{bsmallmatrix}0&0&0\\0&0&0\\0&0&1\end{bsmallmatrix}\}\\
\Rightarrow & \text{Kernel}(A)=\{\begin{bsmallmatrix}1&0&0\\0&0&0\\0&0&0\end{bsmallmatrix},\begin{bsmallmatrix}0&1&0\\1&0&0\\0&0&0\end{bsmallmatrix},\begin{bsmallmatrix}0&0&1\\0&0&0\\1&0&0\end{bsmallmatrix},\begin{bsmallmatrix}0&0&0\\0&1&0\\0&0&0\end{bsmallmatrix},\begin{bsmallmatrix}0&0&0\\0&0&1\\0&1&0\end{bsmallmatrix},\begin{bsmallmatrix}0&0&0\\0&0&0\\0&0&1\end{bsmallmatrix}\}\\
&\text{dim}(A)=6
\end{split}
\end{equation}

\subsubsection{Range}

~

\begin{equation}
\tag{4.2.2}
\begin{split}
&A-A^T\\
=&\begin{bsmallmatrix}0&b-d&c-g\\d-b&0&f-h\\g-c&h-f&0\end{bsmallmatrix}\\
=&\begin{bsmallmatrix}0&1&0\\-1&0&0\\0&0&0\end{bsmallmatrix}(b-d)+\begin{bsmallmatrix}0&0&1\\0&0&0\\-1&0&0\end{bsmallmatrix}(c-g)+\begin{bsmallmatrix}0&0&0\\0&0&1\\0&-1&0\end{bsmallmatrix}(f-h)\\
\Rightarrow &\text{Range}(A)=\{\begin{bsmallmatrix}0&1&0\\-1&0&0\\0&0&0\end{bsmallmatrix},\begin{bsmallmatrix}0&0&1\\0&0&0\\-1&0&0\end{bsmallmatrix},\begin{bsmallmatrix}0&0&0\\0&0&1\\0&-1&0\end{bsmallmatrix}\}\\
&\text{dim}(A)=3\\
\end{split}
\end{equation}

\newpage

\section{Problem 5}

~

\subsection{Question a}

~

\begin{equation}
\tag{5.1}
\begin{split}
&3x^2-x+2\\
=&3(x^2+x-1)-4x+5\\
=&3(x^2+x-1)-4(x+2)+13\\
\Rightarrow & T(3x^2-x+2)\\
=&T(3(x^2+x-1)-4(x+2)+13)\\
=&3T(x^2+x-1)-4T(x+2)+13T(1)\\
=&3\times (2,1)-4\times (1,-1)+13\times T(1)\\
=& (6,3)-(4,-4)+(65,0)\\
=&(67,-1)\\
\end{split}
\end{equation}

\subsection{Question b}

~

\begin{equation}
\tag{5.2}
\begin{split}
&\sum_{i=0}^{n}c_iv_i=0\\
&T(\sum_{i=0}^{n}c_iv_i)\\
=&\sum_{i=0}^{n}c_iT(v_i)\\
=&0\\
&T(v_i) \text{ LI}\\
\Rightarrow &c_i\text{ can only }=0\\
\Rightarrow &v_i \text{ LI}\\
\end{split}
\end{equation}

\subsection{Question c}

~

\begin{equation}
\tag{5.3}
\begin{split}
&T \text{ is one-to-one}\\
\Rightarrow &\text{nullity}(T)=0\\
&\text{nullity}(T)+\text{rank}(T)=\text{dim}(V)\\
\Rightarrow &\text{rank}(T)=\text{dim}(V)\\
\Rightarrow &\text{dim}(\text{Range}(T))=\text{dim}(V)\\
\Rightarrow &\text{Range}(T)=V\\
\Rightarrow &T\text{ is onto}
\end{split}
\end{equation}

\newpage

\section{Reference}

\subsection{Collaborator}

~

Frank Zhu

~

Jeffery Shu
\end{document}