\documentclass{article}
\usepackage[utf8]{inputenc}
\usepackage{setspace}
\usepackage{amssymb}
\usepackage{amsmath}
\usepackage{amsthm}

\def\R{\mathbb{R}}
\DeclareMathOperator{\im}{im}

\begin{document}

\section{Problem 1}

\subsection{Question a}

~

According to the problem, we can have the equations:

\begin{equation*}
\begin{split}
T(x+1)&=\alpha_1 (x^2+x+1)+\beta_1 (x-1)+\gamma_1\\
&=\alpha_1 x^2+(\alpha_1+\beta_1)x+(\alpha_1-\beta_1+\gamma_1)\\
T(x-2)&=\alpha_2 (x^2+x+1)+\beta_2 (x-1)+\gamma_2\\
&=\alpha_2 x^2+(\alpha_2+\beta_2)x+(\alpha_2-\beta_2+\gamma_2)
\end{split}
\end{equation*}

~

Also according to the linear transformation:

\begin{equation*}
\begin{split}
T(x+1)&=x((x+1)+1)+(1+1)+\int^x_0 t+1 dt\\
&=x^2+2x+2+\left[\frac{1}{2}t^2+t\right]^x_0\\
&=x^2+2x+2+\frac{1}{2}x^2+x\\
&=\frac{3}{2}x^2+3x+2\\
T(x-2)&=x((x+1)-2)+(1-2)+\int^x_0 t-2 dt\\
&=x^2-x-1+\left[\frac{1}{2}t^2-2t\right]^x_0\\
&=x^2-x-1+(\frac{1}{2}x^2-2x)\\
&=\frac{3}{2}x^2-3x-1\\
\end{split}
\end{equation*}

From the two sets of equations above, we can see that:

\begin{equation*}
\begin{split}
&\begin{cases}
\alpha_1=\frac{3}{2}\\
\alpha_1+\beta_1=3\\
\alpha_1-\beta_1+\gamma_1=2\\
\end{cases}\\
\Rightarrow &\begin{cases}
\alpha_1=\frac{3}{2}\\
\beta_1=\frac{3}{2}\\
\gamma_1=2\\
\end{cases}\\
&\begin{cases}
\alpha_2=\frac{3}{2}\\
\alpha_2+\beta_2=-3\\
\alpha_2-\beta_2+\gamma_2=-1\\
\end{cases}\\
\Rightarrow &\begin{cases}
\alpha_2=\frac{3}{2}\\
\beta_2=-\frac{9}{2}\\
\gamma_2=-7
\end{cases}\\
\end{split}
\end{equation*}

~

Hence:
$$
\left[T\right]^\gamma_\beta=\begin{bmatrix}
\frac{3}{2}&\frac{3}{2}\\
\frac{3}{2}&-\frac{9}{2}\\
2&-7\\
\end{bmatrix}
$$

\subsection{Question b}

\begin{equation*}
\begin{split}
T(3x-5) =&x(3(x+1)-5)+(3-5)+\int^x_0 3t-5 dt\\
=&3x^2-2x-2+\left[\frac{3}{2}t^2-5t\right]^x_0\\
=&3x^2-2x-2+(\frac{3}{2}x^2-5x)\\
=&\frac{9}{2}x^2-7x-2\\
\Rightarrow &\begin{cases}
\alpha_p=\frac{9}{2}\\
\alpha_p+\beta_p=-7\\
\alpha_p-\beta_p+\gamma_p=-2\\
\end{cases}\\
\Rightarrow &\begin{cases}
\alpha_p=\frac{9}{2}\\
\beta_p=-\frac{23}{2}\\
\gamma_p=-18\\
\end{cases}\\
\Rightarrow &\left[T(3x-5)\right]_\gamma=\begin{bmatrix}
\frac{9}{2}\\
-\frac{23}{2}\\
-18\\
\end{bmatrix}\\
3x-5 =& m(x+1)+n(x-2)\\
=&(m+n)x+(m-2n)\\
\Rightarrow &\begin{cases}
m+n=3\\
m-2n=-5\\
\end{cases}\\
\Rightarrow &\begin{cases}
m=\frac{1}{3}\\
n=\frac{8}{3}\\
\end{cases}\\
\Rightarrow &\left[3x-5\right]_\beta =\begin{bmatrix}
\frac{1}{3}\\
\frac{8}{3}\\
\end{bmatrix}\\
\Rightarrow &\left[T\right]^\gamma_\beta\left[3x-5\right]_\beta\\
=&\begin{bmatrix}
\frac{5}{2}&\frac{3}{2}\\
3&-\frac{9}{2}\\
0&5\\
\end{bmatrix} \begin{bmatrix}
\frac{1}{3}\\
\frac{8}{3}\\
\end{bmatrix}\\
&\left[T\right]^\gamma_\beta\left[p(x)\right]_\beta\\
=&\begin{bmatrix}
\frac{3}{2}&\frac{3}{2}\\
\frac{3}{2}&-\frac{9}{2}\\
2&-7\\
\end{bmatrix}\begin{bmatrix}
\frac{1}{3}\\
\frac{8}{3}\\
\end{bmatrix}\\
=&\begin{bmatrix}
\frac{9}{2}\\
-\frac{23}{2}\\
-18\\
\end{bmatrix}\\
=&\left[T(3x-5)\right]_\gamma\\
\Rightarrow &\left[T(3x-5)\right]_\gamma=\left[T\right]^\gamma_\beta\left[p(x)\right]_\beta\\
\end{split}
\end{equation*}

\newpage

\section{Problem 2}

~

Define $p(x)=ax^2+bx+c$
\subsection{Question a}

~

\begin{equation*}
\begin{split}
T(p(x))&=\frac{1}{x}\int^x_0(a(t+1)^2+b(t+1)+c)dt\\
&=\frac{1}{x}\int^x_0(at^2+2at+a+bt+b+c)dt\\
&=\frac{1}{x}\int^x_0(at^2+(2a+b)t+(a+b+c))dt\\
&=\frac{1}{x}\left[\frac{1}{3}at^3+\frac{1}{2}(2a+b)t^2+(a+b+c)t\right]^x_0\\
&=\frac{1}{x}(\frac{1}{3}ax^3+\frac{1}{2}(2a+b)x^2+(a+b+c)x)\\
&=\frac{1}{3}ax^2+\frac{1}{2}(2a+b)x+(a+b+c)\\
\end{split}
\end{equation*}

\subsubsection{One-to-one}

~

\begin{equation*}
\begin{split}
&\frac{1}{3}ax^2+\frac{1}{2}(2a+b)x+(a+b+c)=0\\
\Rightarrow &\begin{cases}
\frac{1}{3}a=0\\
\frac{1}{2}(2a+b)=0\\
a+b+c=0\\
\end{cases}\\
\Rightarrow &\begin{cases}
a=0\\
b=0\\
c=0\\
\end{cases}\rightarrow \text{trivial solution}\\
\Rightarrow &\text{One-to-one}\\
\end{split}
\end{equation*}

\subsubsection{Onto}

~

\begin{equation*}
\begin{split}
&\begin{bmatrix}
\frac{1}{3}a\\
\frac{1}{2}(2a+b)\\
a+b+c\\
\end{bmatrix}\\
=&\begin{bmatrix}
\frac{1}{3}&0&0\\
1&\frac{1}{2}&0\\
1&1&1\\
\end{bmatrix}\begin{bmatrix}
a\\
b\\
c\\
\end{bmatrix}\\
&\begin{bmatrix}
\frac{1}{3}&0&0\\
1&\frac{1}{2}&0\\
1&1&1\\
\end{bmatrix}\\
\sim & \begin{bmatrix}
1&0&0\\
1&\frac{1}{2}&0\\
1&1&1\\
\end{bmatrix}\\
\sim &\begin{bmatrix}
1&0&0\\
0&1&0\\
1&1&1\\
\end{bmatrix}\\
\sim &\begin{bmatrix}
1&0&0\\
0&1&0\\
0&0&1\\
\end{bmatrix}\\
\end{split}
\end{equation*}

~

Since there is a pivot in every row, $T$ is onto.

\subsection{Question b}

~

$$
\beta=\{1,x,x^2\}
$$

~

Define $\beta_U=\{1,x,x^2\}$ and $\beta_V=\{1,x,x^2\}$ for $p(x)$ and $T(p(x))$ respectively.

\begin{equation*}
\begin{split}
T(1)=&\alpha_1+\beta_1 x+\gamma_1 x^2\\
T(x)=&\alpha_2+\beta_2 x+\gamma_2 x^2\\
T(x^2)=&\alpha_3+\beta_3 x+\gamma_3 x^2\\
\end{split}
\end{equation*}

~

According to the transformation:

\begin{equation*}
\begin{split}
T(1)=& \frac{1}{x}\int^x_0 1dt\\
=&\frac{1}{x}\left[t\right]^x_0\\
=&\frac{1}{x}x\\
=&1\\
T(x)=&\frac{1}{x}\int^x_0 (t+1)dt\\
=&\frac{1}{x}\left[\frac{1}{2}t^2+t\right]^x_0\\
=&\frac{1}{x}(\frac{1}{2}x^2+x)\\
=&\frac{1}{2}x+1\\
T(x^2)=&\frac{1}{x}\int^x_0(t+1)^2dt\\
=&\frac{1}{x}\int^x_0(t^2+2t+1)dt\\
=&\frac{1}{x}\left[\frac{1}{3}t^3+t^2+t\right]^x_0\\
=&\frac{1}{x}(\frac{1}{3}x^3+x^2+x)\\
=&\frac{1}{3}x^2+x+1\\
\end{split}
\end{equation*}

~

From the two sets, we can see that:

\begin{equation*}
\begin{split}
&\begin{cases}
\alpha_1=1\\
\beta_1=0\\
\gamma_1=0\\
\end{cases}\\
&\begin{cases}
\alpha_2=1\\
\beta_2=\frac{1}{2}\\
\gamma_2=0\\
\end{cases}\\
&\begin{cases}
\alpha_3=1\\
\beta_3=1\\
\gamma_3=\frac{1}{3}\\
\end{cases}\\
\Rightarrow &M=\left[T\right]^\beta_\beta\begin{bmatrix}
1&1&1\\
0&\frac{1}{2}&1\\
0&0&\frac{1}{3}\\
\end{bmatrix}\\
\end{split}
\end{equation*}

\subsubsection{Question c}

~

\begin{equation*}
\begin{split}
&\left[\begin{array}{ccc|ccc}
1 & 1 & 1 & 1 & 0 & 0\\
0 & \frac{1}{2} & 1 & 0 & 1 & 0\\
0 & 0 & \frac{1}{3} & 0 & 0 & 1\\
\end{array}\right]\\
\sim &\left[\begin{array}{ccc|ccc}
1 & 1 & 1 & 1 & 0 & 0\\
0 & 1 & 2 & 0 & 2 & 0\\
0 & 0 & 1 & 0 & 0 & 3\\
\end{array}\right]\\
\sim &\left[\begin{array}{ccc|ccc}
1 & 1 & 1 & 1 & 0 & 0\\
0 & 1 & 0 & 0 & 2 & -6\\
0 & 0 & 1 & 0 & 0 & 3\\
\end{array}\right]\\
\sim &\left[\begin{array}{ccc|ccc}
1 & 0 & 0 & 1 & -2 & 3\\
0 & 1 & 0 & 0 & 2 & -6\\
0 & 0 & 1 & 0 & 0 & 3\\
\end{array}\right]\\
\Rightarrow & M^{-1}=\left[\begin{array}{ccc}
1&-2&3\\
0&2&-6\\
0&0&3\\
\end{array}\right]\\
\Rightarrow &\left[T\right]^\beta_\beta=\left[\begin{array}{ccc}
1&-2&3\\
0&2&-6\\
0&0&3\\
\end{array}\right]\\
\Rightarrow &\begin{cases}
T^{-1}(1)=1\\
T^{-1}(x)=2x-2\\
T^{-1}(x^2)=3x^2-6x+3\\
\end{cases}\\
\end{split}
\end{equation*}

So we can calculate $T^{-1}(ax^2+bx+c)$:

\begin{equation*}
\begin{split}
T^{-1}(ax^2+bx+c)=&T^{-1}(ax^2)+T^{-1}(bx)+T^{-1}(c)\\
=&aT^{-1}(x^2)+bT^{-1}(x)+cT^{-1}(1)\\
=&a(3x^2-6x+3)+b(2x-2)+c\\
=&3ax^2+(-6a+2b)x+(3a-2b+c)\\
\end{split}
\end{equation*}

\newpage

\section{Problem 3}
\subsection{Question a}

~

Let $x\in \ker(T)$, where $x$ is arbitrary.

\begin{equation*}
\begin{split}
&T(x)=0_v\\
&S(0_v)=0_w\\
\Rightarrow & S(T(x))=0_w\\
\Rightarrow &x \in \ker(S\circ T)\\
\end{split}
\end{equation*}

~

Since for arbitrary $x\in \ker(T)$ there is $x\in \ker(S\circ T) \Rightarrow \ker(T) \subseteq \ker(S\circ T)$

~

Let $w\in \im (S\circ T)$, where $w$ is arbitrary.

\begin{equation*}
\begin{split}
&\exists u : w=S\circ T(u)=S(T(u))\\
\Rightarrow &w=S(v) \land v=T(u)\\
\Rightarrow &w\in\im(S)\\
\end{split}
\end{equation*}

~

Since for arbitrary $w\in \im (S\circ T)$ there is $w\in \im(S) \Rightarrow \im(S\circ T) \subseteq \im(S)$

\subsection{Question b}

~

According to rank-nullity theorem:
$$
\dim(U)=\dim(\ker(T))+\dim(\im(T))
$$

~

Also:
$$
\dim(U)=\dim(\ker(S\circ T))+\dim(\im(S\circ T))
$$

\begin{equation*}
\begin{split}
\Rightarrow &\dim(\ker(T))+\dim(\im(T))=\dim(\ker(S\circ T))+\dim(\im(S\circ T))\\
\Rightarrow &\dim(\ker(S\circ T))-\dim(\ker(T))=\dim(\im(T))-\dim(\im(S\circ T))\\
\end{split}
\end{equation*}

\subsection{Question c}

~

\begin{equation*}
\tag{conclusion 1}
\begin{split}
\text{Let } &u\in \ker(\hat{S}) \subseteq \hat{V} \subseteq V\\
\Rightarrow & \hat{S}(u)=0_w\\
& \hat{S}(u) =S(u)=0_w\\
\Rightarrow &u\in\ker(S)\\
\text{Also, }&u\in\hat{V}=\im(T)\\
\Rightarrow &u\in \ker(S)\cap\im(T)\\
\Rightarrow &\ker(\hat{S}) \subseteq \ker(S)\cap\im(T) \rightarrow u\text{ is arbitrary in } \ker(\hat{S})\\
\text{Let } & x\in \ker(S)\cap\im(T)\\
\Rightarrow  & x \in \ker(S)\land x\in\im(T)\\
\Rightarrow &S(x)=0_w\land x\in \hat{V}\\
&S(x)=\hat{S}(x)=0_w\\
\Rightarrow &\hat{S}(x)=0_w\land x\in\hat{V}\\
\Rightarrow &x\in\ker(\hat{S})\\
\Rightarrow &\ker(S)\cap\im(T)\subseteq\ker(\hat{S})\\
&\ker(\hat{S}) \subseteq \ker(S)\cap\im(T)\\
\Rightarrow &\ker(S)\cap\im(T)=\ker(\hat{S})
\end{split}
\end{equation*}

~

\begin{equation*}
\tag{conclusion 2}
\begin{split}
\text{Let }&w \in \im(\hat{S})\\
\Rightarrow &\exists m\in \hat{V} =\im(T) : \hat{S}(m)=w\\
&\hat{S}(m)=S(m)=w\\
&m=T(k) \rightarrow m\in \im(T)\\
\Rightarrow &S(T(k))=w\\
\Rightarrow &w\in\im (S\circ T)\\
\Rightarrow &\im(\hat{S}) \subseteq\im (S\circ T)\\
\text{Let } &v\in\im(S\circ T)\\
\Rightarrow &S\circ T(p)=v\\
&T(p)=q\in \hat{V}\\
\Rightarrow &S(q)=w\\
&S(q)=\hat{S}(q)=w\\
\Rightarrow &q\in\im(\hat{S})\\
\Rightarrow &\im(S\circ T)\subseteq\im(\hat{S})\\
&\im(\hat{S}) \subseteq\im (S\circ T)\\
\Rightarrow & \im(\hat{S}) =\im (S\circ T)
\end{split}
\end{equation*}

\begin{equation*}
\dim(\hat{V})=\dim(\ker(\hat{S}))+\dim(\im(\hat{S}))
\end{equation*}

\subsection{Question d}

~

\begin{equation*}
\begin{split}
\ker(S)\cap\im(T)=&\ker(\hat{S})\\
\dim(\ker(S)\cap\im(T))=&\dim(\ker(\hat{S}))\\
\im(\hat{S}) =&\im (S\circ T)\\
\dim(\im(\hat{S}))=&\dim(\im (S\circ T))\\
\dim(\ker(S\circ T))-\dim(\ker(T))=&\dim(\im(T))-\dim(\im(S\circ T))\\
=&\dim(\hat{V})-\dim(\im(\hat{S}))\\
=&\dim(\ker(\hat{S}))\rightarrow\text{rank-nullity theorem}\\
=&\dim(\ker(S)\cap\im(T))\\
\Rightarrow \dim(\ker(S\circ T))-\dim(\ker(T))=&\dim(\ker(S)\cap\im(T))\\
\Rightarrow \dim(\ker(S\circ T))=&\dim(\ker(T))+\dim(\ker(S)\cap\im(T))\\
\end{split}
\end{equation*}

\section{Problem 4}

\subsection{Question a}

~

\begin{equation*}
\begin{split}
T(a,b,c)&=(a,a,c+b-a)\\
T^2(a,b,c)&=T(T(a,b,c))\\
&=T(a,a,c+b-a)\\
&=(a,a,c+b-a+a-a)\\
&=(a,a,c+b-a)\\
&=T(a,b,c)\\
\Rightarrow &T(a,b,c) \text{ is idempotent}\\
\end{split}
\end{equation*}

\subsection{Question b}

~

\begin{proof}

~

Proof by contradiction:

~

Suppose $u\ne 0_V \in \ker(T)\cap\im(T)$

\begin{equation*}
\begin{split}
u\in\ker(T)\Rightarrow & T(u)=0_V\\
u\in\im(T)\Rightarrow & \exists v\in V : T(v)=u\\
T(u)=&T(T(v))=T^2(v)=0_V\\
u\ne0\Rightarrow & T(v)\ne 0\\
\Rightarrow v\notin \ker(T)\land & v\in ker(T^2)\\
\Rightarrow \dim(\ker(T^2))\ne&\dim(\ker(T))\\
T^2=T\Rightarrow & \dim(\ker(T^2))=\dim(\ker(T))\\
\dim(\ker(T^2))=\dim(\ker(T))\land &\dim(\ker(T^2))\ne\dim(\ker(T)) \Rightarrow \bot\\
\Rightarrow \ker(T)\cap&\im(T)=0_V\\
\end{split}
\end{equation*}
\end{proof}

\subsection{Question c}

~

\begin{equation*}
\begin{split}
&T \text{ is one-to-one}\\
\Rightarrow&\dim(\ker{T})=0\\
\Rightarrow&\dim(\im{T})=\dim(V)\\
\Rightarrow&T \text{ is onto}\\
\Rightarrow& T \text{ is inversible}\\
\end{split}
\end{equation*}

Hence define $T^{-1}:V\rightarrow V$ is the inverse of $T$.

\begin{equation*}
\begin{split}
T^2(v)&=T(v)\\
\Rightarrow T\circ T(v)&=T(v)\\
\Rightarrow T^{-1}\circ T\circ T(v)&=T^{-1}\circ T(v)\\
\Rightarrow T(v)&=v\\
\end{split}
\end{equation*}

\newpage

\section{Problem 5}

\subsection{Question a}

~

\begin{equation*}
\begin{split}
\text{Define } T(x)=Ax \land & S(x)=Bx\\
(T+S)(x)=&T(x)+S(x)\\
=&Ax+Bx\\
=&(A+B)x\\
\Rightarrow \im(T+S)=&\text{span}(A+B)\\
\im(T)=&\text{span}(A)\\
\im(S)=&\text{span}(B)\\
\text{Define }M=&\text{span}(A\cap B)=\{v_1,v_2,...,v_r\}\\
\text{Then we can also define: }&\\
\text{span}(A)=A_1+M, A_1=\{u_1,u_2,...,u_m\}\land& \text{ span}(B)=B_1+M, B_1=\{w_1,w_2,...,w_n\}\\
\Rightarrow \text{span}(A+B)=&\{v_1,v_2,...,v_r,u_1,u_2,...,u_m,w_1,w_2,...,w_n\}\\
\{v_1,v_2,...,v_r,u_1,u_2,...,u_m,w_1,w_2,...,w_n\}\subseteq &\{v_1,v_2,...,v_r,u_1,u_2,...,u_m\}+\{v_1,v_2,...,v_r,w_1,w_2,...,w_n\}\\
\Rightarrow \text{span}(A+B)\subseteq &\text{span}(A)+\text{span}(B)\\
\Rightarrow \im(T+S)\subseteq &\im(T)+\im(S)\\
\end{split}
\end{equation*}

\subsection{Question b}

~

\begin{equation*}
\begin{split}
A=\left[a_{mn}\right]\land &B=\left[b_{mn}\right]\land C=\left[c_{lm}\right]\\
C(A+B)=&\left[c_{lm}\right](\left[a_{mn}\right]+\left[b_{mn}\right])\\
=&[c_{lm}]\left[(a_{mn}+b_{mn})\right]\\
=&\left[\sum^k_{m=1}c_{lk}(a_{kn}+b_{kn})\right]\\
=&\left[\sum^k_{m=1}c_{lk}a_{kn}+\sum^k_{m=1}c_{lk}b_{kn}\right]\\
=&\left[\sum^k_{m=1}c_{lk}a_{kn}\right]+\left[\sum^k_{m=1}c_{lk}b_{kn}\right]\\
=&\left[c_{lm}a_{mn}\right]+\left[c_{lm}b_{mn}\right]\\
=&CA+BA\\
\end{split}
\end{equation*}

\newpage

\section{Reference}

\subsection{Collaborators}

~

Frank Zhu

~

Jeffery Shu
\end{document}