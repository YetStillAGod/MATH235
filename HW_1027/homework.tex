\documentclass{article}
\usepackage[utf8]{inputenc}
\usepackage{setspace}
\usepackage{amssymb}
\usepackage{amsmath}
\usepackage{amsthm}
\usepackage{systeme}
\usepackage{mathtools}

\def\R{\mathbb{R}}
\def\rank{\text{rank}}
\DeclareMathOperator{\im}{im}

\begin{document}
\section{Problem 1}

\subsection{Question a}

~

\subsubsection{$\rank(A)=0\implies A\text{ is a zero matrix}$}

~

\begin{equation*}
\begin{split}
\text{Premise: }&\rank(A)=0\\
&\text{Suppose }A\text{ is not a zero matrix}\\
\Rightarrow &\text{There is at least a non-zero column for the reduced row-echelon form of }A\\
\Rightarrow &\rank(A)\ne0\\
&\rank(A)\ne0\Rightarrow\!\Leftarrow\rank(A)=0\\
\Rightarrow &\rank(A)=0\implies A\text{ is a zero matrix}\\
\end{split}
\end{equation*}

\subsubsection{$A\text{ is a zero matrix}\implies\rank(A)=0$}

~

\begin{equation*}
\begin{split}
\text{Premise: }&A\text{ is a zero matrix}\\
\Rightarrow&\text{There is no non-zero columns or rows in }A\\
\Rightarrow&\rank(A)=0\\
\Rightarrow&A\text{ is a zero matrix}\implies\rank(A)=0\\
\end{split}
\end{equation*}

\subsubsection{Conclusion}

~

$$
\rank(A)=0\iff A\text{ is a zero matrix}
$$

\subsection{Question b}

~

\begin{equation*}
\begin{split}
&\rank(cA)\\
=&\rank(cI_mA)\\
=&\rank((cI_m)A)\\
&cI_m\text{ is invertible since }I_m\text{ is invertible}\\
\Rightarrow&\rank((cI_m)A)=\rank(A)\\
\Rightarrow&\rank(cA)=\rank(A)\\
\end{split}
\end{equation*}

\newpage

\section{Problem 2}

\subsection{Question a}

~

\begin{equation*}
\begin{split}
&\dim(\im(S)+\im(T))\leq\dim(\im(S))+\dim(\im(T))\\
\rightarrow&\text{There can be common elements between }\im(S)\text{ and }\im(T)\\
&\text{It is easy to see that }\im(S+T)\subseteq\im(S)+\im(T)\\
\rightarrow&\text{ Because of the similar reason above}\\
&\im(S+T)\subseteq\im(S)+\im(T)\\
\Rightarrow &\dim(\im(S+T))\leq\dim(\im(S)+\im(T))\\
\Rightarrow&\dim(\im(S+T))\leq\dim(\im(S))+\dim(\im(T))\\
\end{split}
\end{equation*}

\subsection{Question b}

~

\begin{equation*}
\begin{split}
&S\coloneqq Ax\land T\coloneqq Bx\land S+T=(A+B)x\\
\Rightarrow &\dim(\im(S))=\rank(A)\land\dim(\im(T))=\rank(B)\land\dim(\im(S+T))=\rank(A+B)\\
\Rightarrow &\rank(A+B)\leq\rank(A)+\rank(B)\\
\end{split}
\end{equation*}

\newpage

\section{Problem 3}

\subsection{Question a}

~

\begin{equation*}
\begin{split}
&\begin{bmatrix}
0&1&0\\
2&1&0\\
1&0&1\\
\end{bmatrix}\\
\sim&\begin{bmatrix}
2&1&0\\
0&1&0\\
1&0&1\\
\end{bmatrix} S_{12}\\
\sim&\begin{bmatrix}
2&0&0\\
0&1&0\\
1&0&1\\
\end{bmatrix} A_{12}(-1)\\
\sim&\begin{bmatrix}
1&0&0\\
0&1&0\\
1&0&1\\
\end{bmatrix} M_{1}(\frac{1}{2})\\
\sim&\begin{bmatrix}
1&0&0\\
0&1&0\\
0&0&1\\
\end{bmatrix} A_{31}(-1)\\
&\text{The row-echelon form has full-rank, meaning it is invertible}\\
\end{split}
\end{equation*}

\subsection{Question b}

~

\begin{equation*}
\begin{split}
&\left[\begin{array}{ccc|ccc}
0 & 1 & 0 & 1 & 0 & 0\\
2 & 1 & 0 & 0 & 1 & 0\\
1 & 0 & 1 & 0 & 0 & 1\\
\end{array}\right]\\
\sim&\left[\begin{array}{ccc|ccc}
2 & 1 & 0 & 0 & 1 & 0\\
0 & 1 & 0 & 1 & 0 & 0\\
1 & 0 & 1 & 0 & 0 & 1\\
\end{array}\right] S_{12}\\
\sim&\left[\begin{array}{ccc|ccc}
2 & 0 & 0 & -1 & 1 & 0\\
0 & 1 & 0 & 1 & 0 & 0\\
1 & 0 & 1 & 0 & 0 & 1\\
\end{array}\right] A_{12}(-1)\\
\sim&\left[\begin{array}{ccc|ccc}
1 & 0 & 0 & -\frac{1}{2} & \frac{1}{2} & 0\\
0 & 1 & 0 & 1 & 0 & 0\\
1 & 0 & 1 & 0 & 0 & 1\\
\end{array}\right] M_{1}(\frac{1}{2})\\
\sim&\left[\begin{array}{ccc|ccc}
1 & 0 & 0 & -\frac{1}{2} & \frac{1}{2} & 0\\
0 & 1 & 0 & 1 & 0 & 0\\
0 & 0 & 1 & \frac{1}{2} & -\frac{1}{2} & 1\\
\end{array}\right] A_{31}(-1)\\
\end{split}
\end{equation*}

\subsection{Question c}

~

\begin{equation*}
\begin{split}
&S_{12}A_{12}(-1)M_{1}(\frac{1}{2})A_{31}(-1)A=I_{3}\\
\Rightarrow&A^{-1}=S_{12}A_{12}(-1)M_{1}(\frac{1}{2})A_{31}(-1)\\
&A=(A^{-1})^{-1}=(S_{12}A_{12}(-1)M_{1}(\frac{1}{2})A_{31}(-1))^{-1}\\
=&A_{31}(-1)^{-1}M_{1}(\frac{1}{2})^{-1}A_{12}(-1)^{-1}{S_{12}}^{-1}\\
&S_{12}=\begin{bmatrix}
0&1&0\\
1&0&0\\
0&0&1\\
\end{bmatrix} A_{12}(-1)=\begin{bmatrix}
1&-1&0\\
0&1&0\\
0&0&1\\
\end{bmatrix} M_{1}(\frac{1}{2})=\begin{bmatrix}
\frac{1}{2}&0&0\\
0&1&0\\
0&0&1\\
\end{bmatrix} A_{31}(-1)=\begin{bmatrix}
1&0&0\\
0&1&0\\
-1&0&1\\
\end{bmatrix}\\
\end{split}
\end{equation*}

\newpage

\section{Problem 4}

\subsection{Question a}

~

\begin{equation*}
\begin{split}
&\begin{bmatrix}
1&0&-1&2&1\\
-1&1&3&-1&0\\
-2&1&4&-1&3\\
3&-1&-5&1&-6\\
\end{bmatrix}\\
\sim&\begin{bmatrix}
1&0&-1&2&1\\
0&1&2&1&1\\
-2&1&4&-1&3\\
3&-1&-5&1&-6\\
\end{bmatrix}A_{21}(1)\\
\sim&\begin{bmatrix}
1&0&-1&2&1\\
0&1&2&1&1\\
0&1&2&3&5\\
3&-1&-5&1&-6\\
\end{bmatrix}A_{31}(2)\\
\sim&\begin{bmatrix}
1&0&-1&2&1\\
0&1&2&1&1\\
0&1&2&3&5\\
0&-1&-2&-5&-9\\
\end{bmatrix}A_{41}(-3)\\
\sim&\begin{bmatrix}
1&0&-1&2&1\\
0&1&2&1&1\\
0&0&0&2&4\\
0&-1&-2&-5&-9\\
\end{bmatrix}A_{32}(-1)\\
\sim&\begin{bmatrix}
1&0&-1&2&1\\
0&1&2&1&1\\
0&0&0&2&4\\
0&0&0&-4&-8\\
\end{bmatrix}A_{42}(1)\\
\sim&\begin{bmatrix}
1&0&-1&2&1\\
0&1&2&1&1\\
0&0&0&1&2\\
0&0&0&-4&-8\\
\end{bmatrix}M_{3}(\frac{1}{2})\\
\end{split}
\end{equation*}

\begin{equation*}
\begin{split}
\sim&\begin{bmatrix}
1&0&-1&0&-3\\
0&1&2&1&1\\
0&0&0&1&2\\
0&0&0&-4&-8\\
\end{bmatrix}A_{13}(-2)\\
\sim&\begin{bmatrix}
1&0&-1&0&-3\\
0&1&2&0&-1\\
0&0&0&1&2\\
0&0&0&-4&-8\\
\end{bmatrix}A_{23}(-1)\\
\sim&\begin{bmatrix}
1&0&-1&0&-3\\
0&1&2&0&-1\\
0&0&0&1&2\\
0&0&0&0&0\\
\end{bmatrix}A_{43}(4)\\
\end{split}
\end{equation*}

\begin{equation*}
\begin{split}
\Rightarrow &\text{nullspace}(A)=\begin{bmatrix}
s+3t\\
-2s+t\\
s\\
-2t\\
t\\
\end{bmatrix}\\
=&\begin{bmatrix}
1\\
-2\\
1\\
0\\
0\\
\end{bmatrix}s+\begin{bmatrix}
3\\
1\\
0\\
-2\\
1\\
\end{bmatrix}t\\
\Rightarrow &M=\begin{bmatrix}
1&3&0&0&0\\
-2&1&0&0&0\\
1&0&0&0&0\\
0&-2&0&0&0\\
0&1&0&0&0\\
\end{bmatrix}\\
&\dim(\text{col}(M))=2\\
\Rightarrow&\rank(M)=2\\
\end{split}
\end{equation*}

\subsection{Question b}

~

\begin{equation*}
\begin{split}
&AB=0_{4\times5}\\
\Rightarrow&\text{col}(B)\subset\text{nullspace}(A)\\
\Rightarrow&\rank(B)\leq\dim(\text{nullspace}(A))\\
&\dim(\text{nullspace}(A))+\rank(A)=5\rightarrow\text{rank-nullity theorem}\\
&\rank(A)=3\rightarrow\text{There is 3 pivot columns in row-echelon form}\\
\Rightarrow&\dim(\text{nullspace}(A))=2\\
\Rightarrow&\rank(B)\leq2\\
\end{split}
\end{equation*}

\newpage

\section{Problem 5}

~

\begin{equation*}
\begin{split}
&\text{Denote the first to the sixth columns } c_1,...,c_6\\
&\text{Since }a_1,a_3,a_5\text{ are the pivot columns, we can use the three columns to represent the others.}\\
&\begin{cases}
c_2=-3c_1\\
c_4=4c_1+3c_3\\
c_6=5c_1+2c_3-c_5\\
\end{cases}\\
\Rightarrow&\begin{cases}
c_2=\begin{bmatrix}
3\\
-6\\
-3\\
9\\
\end{bmatrix}\\
c_4=\begin{bmatrix}
1\\
-5\\
2\\
0\\
\end{bmatrix}\\
c_6=\begin{bmatrix}
0\\
1\\
-3\\
2\\
\end{bmatrix}\\
\end{cases}\\
\Rightarrow&A=\begin{bmatrix}
1&3&-1&1&3&0\\
-2&-6&1&-5&-9&1\\
-1&-3&2&2&2&-3\\
3&9&-4&0&5&2\\
\end{bmatrix}\\
\end{split}
\end{equation*}

\newpage

\section{Reference}

\subsection{Collaborators}

~

Frank Zhu

~

Jeffery Shu
\end{document}