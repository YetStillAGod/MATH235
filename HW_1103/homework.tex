\documentclass{article}
\usepackage[utf8]{inputenc}
\usepackage{setspace}
\usepackage{amssymb}
\usepackage{amsmath}
\usepackage{amsthm}
\usepackage{systeme}
\usepackage{mathtools}

\def\R{\mathbb{R}}
\def\rank{\text{rank}}
\DeclareMathOperator{\im}{im}

\begin{document}

\section{Problem 1}

\subsection{Question a}

\subsubsection{$(3,-4,7)$}

~

\begin{equation*}
    \begin{split}
        \text{Suppose: }&(3,-4,7)\in \im(T)\\
        \Rightarrow &T(a_1,a_2,a_3)=(3,-4,7)\\
        \Rightarrow &\begin{cases}
            a_1+a_2=3\\
            a_1-a_2-a_3=-4\\
            2a_2+a_3=7\\
        \end{cases}\\
        \Rightarrow&\begin{bmatrix}
            1&1&0\\
            1&-1&-1\\
            0&2&1\\
        \end{bmatrix}\begin{bmatrix}
            a_1\\
            a_2\\
            a_3\\
        \end{bmatrix}=\begin{bmatrix}
            3\\
            -4\\
            7\\
        \end{bmatrix}\\
        \Rightarrow&\left[\begin{array}{ccc|c}
            1&1&0&3\\
            1&-1&-1&-4\\
            0&2&1&7\\
        \end{array}\right]\\
        \sim&\left[\begin{array}{ccc|c}
            1&1&0&3\\
            0&-2&-1&-7\\
            0&2&1&7\\
        \end{array}\right]A_{21}(-1)\\
        \sim&\left[\begin{array}{ccc|c}
            1&1&0&3\\
            0&-2&-1&-7\\
            0&0&0&0\\
        \end{array}\right]A_{32}(1)\\
        \sim&\left[\begin{array}{ccc|c}
            1&1&0&3\\
            0&1&\frac{1}{2}&\frac{7}{2}\\
            0&0&0&0\\
        \end{array}\right]M_2(-\frac{1}{2})\\
        \sim&\left[\begin{array}{ccc|c}
            1&0&-\frac{1}{2}&-\frac{1}{2}\\
            0&1&\frac{1}{2}&\frac{7}{2}\\
            0&0&0&0\\
            \end{array}\right]A_{12}(-1)\\
        \Rightarrow&\begin{cases}
            a_1=\frac{1}{2}a_3-\frac{1}{2}\\
            a_2=-\frac{1}{2}a_3+\frac{7}{2}\\
            a_3=a_3\\
        \end{cases}\\
        \Rightarrow&\text{There are solutions to the system}\\
        \Rightarrow&(3,-4,7)\in\im(T)\\
    \end{split}
\end{equation*}

\subsubsection{$(0,2,2)$}

~

\begin{equation*}
    \begin{split}
        \text{Suppose: }&(0,2,2)\in\im(T)\\
        \Rightarrow&T(b_1,b_2,b_3)=(0,2,2)\\
        \Rightarrow&\begin{cases}
            b_1+b_2=0\\
            b_1-b_2-b_3=2\\
            2b_2+b_3=2\\
        \end{cases}\\
        \Rightarrow&\begin{bmatrix}
            1&1&0\\
            1&-1&-1\\
            0&2&1\\
        \end{bmatrix}\begin{bmatrix}
            b_1\\
            b_2\\
            b_3\\
        \end{bmatrix}=\begin{bmatrix}
            0\\
            2\\
            2\\
        \end{bmatrix}\\
        \Rightarrow&\left[\begin{array}{ccc|c}
            1&1&0&0\\
            1&-1&-1&2\\
            0&2&1&2\\
        \end{array}\right]\\
        \sim&\left[\begin{array}{ccc|c}
            1&1&0&0\\
            0&-2&-1&2\\
            0&2&1&2\\
        \end{array}\right]A_{21}(-1)\\
        \sim&\left[\begin{array}{ccc|c}
            1&1&0&0\\
            0&-2&-1&2\\
            0&0&0&4\\
        \end{array}\right]A_{32}(1)\\
        \Rightarrow&\text{The final row implies that }0=4\Rightarrow\!\Leftarrow\\
        \Rightarrow&(0,2,2)\notin\im(T)\\
    \end{split}
\end{equation*}

\subsection{Question b}

~

\begin{equation*}
    \begin{split}
        &x_1+3x_2+x_3-2x_4=0\\
        \Rightarrow&x_1=-3x_2-x_3+2x_4\\
        \Rightarrow&\begin{bmatrix}
            x_1\\
            x_2\\
            x_3\\
            x_4\\
        \end{bmatrix}=\begin{bmatrix}
            -3x_2-x_3+2x_4\\
            x_2\\
            x_3\\
            x_4\\
        \end{bmatrix}\\
        \Rightarrow&x_2\begin{bmatrix}
            -3\\
            1\\
            0\\
            0\\
        \end{bmatrix}+x_3\begin{bmatrix}
            -1\\
            0\\
            1\\
            0\\
        \end{bmatrix}+x_4\begin{bmatrix}
            2\\
            0\\
            0\\
            1\\
        \end{bmatrix}\\
        &\text{It is easy to see that the three matrices are linear independent since is a unique value in every matrix}\\
        \Rightarrow&\text{span}(B)=\left\{\begin{bmatrix}
            -3\\
            1\\
            0\\
            0\\
        \end{bmatrix},\begin{bmatrix}
            -1\\
            0\\
            1\\
            0\\
        \end{bmatrix},\begin{bmatrix}
            2\\
            0\\
            0\\
            1\\
        \end{bmatrix}\right\}\\
        \Rightarrow&a_1\begin{bmatrix}
            -3\\
            1\\
            0\\
            0\\
        \end{bmatrix}+a_2\begin{bmatrix}
            -1\\
            0\\
            1\\
            0\\
        \end{bmatrix}+a_3\begin{bmatrix}
            2\\
            0\\
            0\\
            1\\
        \end{bmatrix}=\begin{bmatrix}
            1\\
            2\\
            -1\\
            3\\
        \end{bmatrix}\\
        \Rightarrow&\begin{bmatrix}
            -3a_1-a_2+2a_3\\
            a_1\\
            a_2\\
            a_3\\
        \end{bmatrix}=\begin{bmatrix}
            1\\
            2\\
            -1\\
            3\\
        \end{bmatrix}\\
        \Rightarrow&\begin{cases}
            a_1=2\\
            a_2=-1\\
            a_3=3\\
        \end{cases}\\
        \Rightarrow&\begin{bmatrix}
            1\\
            2\\
            -1\\
            3\\
        \end{bmatrix}\text{ can be obtained by linear combinations of the other three}\\
        &\text{and is not a scalar multiplication of one of the matrices}\\
        \Rightarrow&\text{the four matrices can form the basis by either three in the four}\\
        \Rightarrow&\text{span}(B)=\left\{\begin{bmatrix}
            -3\\
            1\\
            0\\
            0\\
        \end{bmatrix},\begin{bmatrix}
            -1\\
            0\\
            1\\
            0\\
        \end{bmatrix},\begin{bmatrix}
            1\\
            2\\
            -1\\
            3\\
        \end{bmatrix}\right\}\\
    \end{split}
\end{equation*}

\newpage

\section{Problem 2}

\subsection{Zero Matrix}

~

\begin{equation*}
    \begin{split}
        &\overrightarrow{0}\overrightarrow{v}=\overrightarrow{0}\\ 
        \Rightarrow &\overrightarrow{0} \in W\\
    \end{split}
\end{equation*}

\subsection{Addition}

~

\begin{equation*}
    \begin{split}
        &\exists X_1\land X_2\in W\\
        &(X_1+X_2)\overrightarrow{v}\\
        =&X_1\overrightarrow{v}+X_2\overrightarrow{v}\\
        =&\overrightarrow{0}+\overrightarrow{0}\\
        =&\overrightarrow{0}\\
        \Rightarrow&X_1+X_2\in W\\
        \end{split}
\end{equation*}

\subsection{Scalar multiplication}

~

\begin{equation*}
    \begin{split}
        &\exists X_1\in W\land a\in\R \\
        &(aX_1)\overrightarrow{v}\\
        =&a(X_1 \overrightarrow{v})\\
        =&a\cdot 0\\
        =&0\\
        \Rightarrow aX_1\in W\\
    \end{split}
\end{equation*}

\subsection{Conclusion and Dimension}

~

Since $W$ meets all the requirements of a subspace, $W$ is a subspace of $\R$

\begin{equation*}
    \begin{split}
        &X\coloneqq \begin{bmatrix}
            a_{11}&a_{12}&a_{13}\\
            a_{21}&a_{22}&a_{23}\\
        \end{bmatrix}\\
        &X\overrightarrow{v}=\overrightarrow{0}\\
        &\begin{bmatrix}
            a_{11}&a_{12}&a_{13}\\
            a_{21}&a_{22}&a_{23}\\
        \end{bmatrix}\begin{bmatrix}
            1\\
            -1\\
            2\\
        \end{bmatrix}=\overrightarrow{0}\\
        \Rightarrow &\begin{bmatrix}
            a_{11}\\
            a_{21}\\
        \end{bmatrix}-\begin{bmatrix}
            a_{12}\\
            a_{22}\\
        \end{bmatrix}+2\begin{bmatrix}
            a_{13}\\
            a_{23}\\
        \end{bmatrix}=\overrightarrow{0}\\
        &s\coloneqq\begin{bmatrix}
            a_{12}\\
            a_{22}\\
        \end{bmatrix}\land t\coloneqq\begin{bmatrix}
            a_{13}\\
            a_{23}\\
        \end{bmatrix}\\
        \Rightarrow &\begin{bmatrix}
            a_{11}\\
            a_{21}\\
        \end{bmatrix}=s-2t\\
        \Rightarrow&X=\left[s-2t,s,t\right]\\
        &X=[1,1,0]s+[-2,0,1]t\\
        &\text{It is easy to see that the two matrices are linear independent}\\
        \Rightarrow &\text{basis}(W)=\left\{[1,1,0],[-2,0,1]\right\}\\
        \Rightarrow &\dim(W)=2\\
    \end{split}
\end{equation*}

\newpage

\section{Problem 3}

\subsection{Question a}

\subsubsection{$\rank(A)=2\implies b\ne0$}

~

\begin{equation*}
    \begin{split}
        &A=\begin{bmatrix}
            a_{11}&a_{12}&a_{13}\\
            a_{21}&a_{22}&a_{23}\\
        \end{bmatrix}\\
        &\rank(A)=2\\
        \Rightarrow &\text{There must be 2 linear independent columns}\\
        \Rightarrow &b_1,b_2,b_3\text{ must have one }\ne 0\\
        \Rightarrow &b\ne0\\
    \end{split}
\end{equation*}

\subsubsection{$b\ne0\implies \rank(A)=2$}

~

\begin{equation*}
    \begin{split}
        &b\ne0\\
        \Rightarrow &\text{At least one of }b_1,b_2,b_3\ne0\\
        &{Suppose }b_1\ne0\\
        \Rightarrow&\begin{bmatrix}
            a_{12}\\
            a_{22}\\
        \end{bmatrix}
        ,\begin{bmatrix}
            a_{13}\\
            a_{23}\\
        \end{bmatrix}\text{ Linear Independent}\\
        \Rightarrow&\left\{\begin{bmatrix}
            a_{12}\\
            a_{22}\\
        \end{bmatrix}\begin{bmatrix}
            a_{13}\\
            a_{23}\\
        \end{bmatrix}\right\} \text{ is a basis of }A\\
        \Rightarrow &\rank(A)=2\\
    \end{split}
\end{equation*}

\subsubsection{Conclusion}

~

\begin{equation*}
    \rank(A)=2\Leftrightarrow b\ne0
\end{equation*}

\subsection{Question b}

~

\begin{equation*}
    \begin{split}
        &\begin{cases}
            b_1=a_{12}a_{23}-a_{22}a_{13}\\
            b_2=a_{13}a_{21}-a_{11}a_{23}\\
            b_3=a_{11}a_{22}-a_{12}a_{21}\\
        \end{cases}\\
        &Ax=0\\
        \Rightarrow &\left[\begin{array}{ccc|c}
            a_{11}&a_{12}&a_{13}&0\\
            a_{21}&a_{22}&a_{23}&0\\
        \end{array}\right]\\
        \sim&\left[\begin{array}{ccc|c}
            1&\frac{a_{12}}{a_{11}}&\frac{a_{13}}{a_{11}}&0\\
            a_{21}&a_{22}&a_{23}&0\\
        \end{array}\right]M_1(\frac{1}{a_{11}})\\
        \sim&\left[\begin{array}{ccc|c}
            1&\frac{a_{12}}{a_{11}}&\frac{a_{13}}{a_{11}}&0\\
            0&\frac{a_{11}a_{22}-a_{12}a_{21}}{a_{11}}&\frac{a_{23}a_{11}-a_{13}a_{21}}{a_{11}}&0\\
        \end{array}\right]A_{12}(-a_{21})\\
        \sim&\left[\begin{array}{ccc|c}
            1&\frac{a_{12}}{a_{11}}&\frac{a_{13}}{a_{11}}&0\\
            0&1&\frac{a_{23}a_{11}-a_{13}a_{21}}{a_{11}a_{22}-a_{12}a_{21}}&0\\
        \end{array}\right]M_2(\frac{a_{11}}{a_{11}a_{22}-a_{12}a_{21}})\\
        \sim&\left[\begin{array}{ccc|c}
            1&0&\frac{a_{13}(a_{11}a_{22}-a_{12}a_{21})-a_{12}(a_{23}a_{11}-a_{13}a_{21})}{a_{11}(a_{11}a_{22}-a_{12}a_{21})}&0\\
            0&1&\frac{a_{23}a_{11}-a_{13}a_{21}}{a_{11}a_{22}-a_{12}a_{21}}&0\\
        \end{array}\right]A_{21}(-\frac{a_{12}}{a_{11}})\\
        =&\left[\begin{array}{ccc|c}
            1&0&\frac{-a_{12}a_{23}+a_{22}a_{13}}{a_{11}a_{22}-a_{12}a_{21}}&0\\
            0&1&\frac{a_{23}a_{11}-a_{13}a_{21}}{a_{11}a_{22}-a_{12}a_{21}}&0\\
        \end{array}\right]\\
        =&\left[\begin{array}{ccc|c}
            1&0&\frac{-b_1}{b_3}&0\\
            0&1&\frac{-b_2}{b_3}&0\\
        \end{array}\right]\\
        &\text{Set col}(3)=b_3t\\
        \Rightarrow&\begin{cases}
            \text{col}(1)=b_1t\\
            \text{col}(2)=b_2t\\
            \text{col}(3)=b_3t\\
        \end{cases}\\
        \Rightarrow &\text{nullspace}(A)=(b_1,b_2,b_3)=b\\
    \end{split}
\end{equation*}

\newpage

\section{Problem 4}

\subsection{Question a}

~

\begin{equation*}
    \begin{split}
        A^T&=-A\\
        \det(A^T)&=\det(-A)\\
        \det(A)&=(-1)^n\det(A)\\
        \det(A)+(-1)^{n+1}\det(A)&=0\\
        \det(A)=0&\Leftrightarrow n=2k+1,k\in\mathbb{Z}^+\\
    \end{split}
\end{equation*}

\subsection{Question b}

~

\begin{equation*}
    \begin{split}
        A^{-1}&=A^T\\
        \det(A^{-1})&=\det(A^T)\\
        \frac{1}{\det(A)}&=\det(A)\\
        (\det(A))^2&=1\\
        \det(A)&=\pm 1
    \end{split}
\end{equation*}

\subsection{Question c}

~

\begin{equation*}
    \begin{split}
        &\begin{vmatrix}
            a&b\\
            c&d\\
        \end{vmatrix}=ad-bc=-2\\
        &\begin{vmatrix}
            3d&3c-6d\\
            b+2d&a-2b+2c-4d\\
        \end{vmatrix}\\
        =&3d(a-2b+2c-4d)-(b+2d)(3c-6d)\\
        =&3ad-6bd+6bc-12d^2-(3bc-6bd+6cd-12d^2)\\
        =&3ad-3bc\\
        =&3(ad-bc)\\
        =&-6
    \end{split}
\end{equation*}

\newpage

\section{Problem 5}

\subsection{Question a}

~

\begin{equation*}
    \begin{split}
        &\beta =\left\{x^2,x,1\right\}\\
        &\gamma=\left\{(1,0,0),(0,1,0),(0,0,1)\right\}\\
        T(1)&=(1,1,1)\\
        &=1(1,0,0)+1(0,1,0)+1(0,0,1)\\
        T(x)&=(a,b,c)\\
        &=a(1,0,0)+b(0,1,0)+c(0,0,1)\\
        T(x^2)&=(a^2,b^2,c^2)\\
        &=a^2(1,0,0)+b^2(0,1,0)+c^2(0,0,1)\\
        \Rightarrow&A=\left[T\right]^\gamma_\beta=\begin{bmatrix}
            a^2&a&1\\
            b^2&b&1\\
            c^2&c&1\\
        \end{bmatrix}
    \end{split}
\end{equation*}

\subsection{Question b}

\subsubsection{$\det(A)\ne0\implies a\ne b\ne c$}

~

\begin{equation*}
    \begin{split}
        &\text{Suppose }b=c\\
        \Rightarrow&A=\begin{bmatrix}
            a^2&a&1\\
            b^2&b&1\\
            b^2&b&1\\
        \end{bmatrix}\\
        &\det(A)=\begin{vmatrix}
            a^2&a&1\\
            b^2&b&1\\
            b^2&b&1\\
        \end{vmatrix}\\
        &=\begin{vmatrix}
            a^2&a&1\\
            b^2&b&1\\
            0&0&0\\
        \end{vmatrix}=0\\
        &\text{The cases of } a=b \lor a=c \text{ is the same for }\det(A)\\
        \Rightarrow& a=b\lor b=c\lor a=c\implies \det(A)=0\\
        &\text{Since }a=b\lor b=c\lor a=c\implies \det(A)=0 \\
        &\text{ and }\det(A)\ne0\implies a\ne b\ne c\text{ are contrapositive}\\
        \Rightarrow &\det(A)\ne0\implies a\ne b\ne c\\
    \end{split}
\end{equation*}

\subsubsection{$a\ne b\ne c\implies\det(A)\ne0$}

~

\begin{equation*}
    \begin{split}
        &\text{Suppose }\det(A)=0\\
        &\begin{vmatrix}
            a^2&a&1\\
            b^2&b&1\\
            c^2&c&1\\
        \end{vmatrix}=0\\
        &\begin{vmatrix}
            a^2&a&1\\
            b^2&b&1\\
            c^2-b^2&c-b&0\\
        \end{vmatrix}=0\\
        &b^2(c-b)+a(c^2-b^2)-(b(c^2-b^2)-a^2(c-b))=0\\
        &(b^2-a^2)(c-b)+(a-b)(c^2-b^2)=0\\
        &(b+a)(b-a)(c-b)+(a-b)(c+b)(c-b)=0\\
        &(c-b)(b-a)(a-c)=0\\
        &b=c\lor a=b\lor a=c\\
        &\det(A)=0 \implies b=c\lor a=b\lor a=c\\
        &\text{Since }\det(A)=0 \implies b=c\lor a=b\lor a=c \\
        &\text{ and }a\ne b\ne c\implies\det(A)\ne0\text{ are contrapositive}\\
        \Rightarrow&a\ne b\ne c\implies\det(A)\ne0\\
    \end{split}
\end{equation*}

\subsubsection{Conclusion}

\begin{equation*}
    \det(A)\ne0\Leftrightarrow a\ne b\ne c
\end{equation*}

\newpage

\section{Reference}

\subsection{Collaborators}

~

Frank Zhu

~

Jeffery Shu
\end{document}