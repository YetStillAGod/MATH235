\documentclass{article}
\usepackage[utf8]{inputenc}
\usepackage{setspace}
\usepackage{amssymb}
\usepackage{amsmath}
\usepackage{amsthm}
\usepackage{systeme}
\usepackage{mathtools}
\usepackage{hyperref}

\def\R{\mathbb{R}}
\def\rank{\text{rank}}
\DeclareMathOperator{\im}{im}

\begin{document}

\section{Problem 1}

\subsection{Question a}

\subsubsection{rref}

~

\begin{equation*}
    \begin{split}
        &\begin{bmatrix}
            1&0&2\\
            -1&2&0\\
            1&0&3\\
        \end{bmatrix}\\
        \sim&\begin{bmatrix}
            1&0&2\\
            0&2&2\\
            1&0&3\\
        \end{bmatrix}A_{12}(1)\\
        \sim&\begin{bmatrix}
            1&0&2\\
            0&1&1\\
            1&0&3\\
        \end{bmatrix}M_{2}(\frac{1}{2})\\
        \sim&\begin{bmatrix}
            1&0&2\\
            0&1&1\\
            0&0&1\\
        \end{bmatrix}A_{13}(-1)\\
        \sim&\begin{bmatrix}
            1&0&0\\
            0&1&1\\
            0&0&1\\
        \end{bmatrix}A_{31}(-2)\\
        \sim&\begin{bmatrix}
            1&0&0\\
            0&1&0\\
            0&0&1\\
        \end{bmatrix}A_{32}(-1)\\
        &A\text{ can be reduced to }I\implies A\text{ is invertible}.\\
        \Rightarrow &A_{32}(-1)A_{31}(-2)A_{13}(-1)M_{2}(\frac{1}{2})A_{12}(1)A=I\\
        \Rightarrow &A^{-1}=A_{32}(-1)A_{31}(-2)A_{13}(-1)M_{2}(\frac{1}{2})A_{12}(1)\\
        \Rightarrow&A^{-1}=\begin{bmatrix}
            3&0&-2\\
            \frac{3}{2}&\frac{1}{2}&-1\\
            -1&0&1\\
        \end{bmatrix}\\
    \end{split}
\end{equation*}

\subsubsection{Caley-Hamilton Theorem}

~

\begin{equation*}
    \begin{split}
        &p_A(\lambda)=\det(\lambda I-A)\\
        \Rightarrow&p_A(\lambda)=\lambda^3-6\lambda^2+9\lambda-2\\
        \Rightarrow&p_A(A)=A^3-6A^2+9A-2I=0\\
        &A^{-1}=\frac{1}{2}(A^2-6A+9I)\\
        \Rightarrow& A^{-1}=\begin{bmatrix}
            3&0&-2\\
            \frac{3}{2}&\frac{1}{2}&-1\\
            -1&0&1\\
        \end{bmatrix}\\
    \end{split}
\end{equation*}

\subsection{Question b}

~

\begin{equation*}
    \begin{split}
        &\beta=\{1,x,x^2\}\\
        &T(1)=2\\
        &T(x)=x+1\\
        &T(x^2)=(x-1)^2+4=x^2-2x+5\\
        \Rightarrow&A=\left[T\right]^\beta_\beta=\begin{bmatrix}
            2&1&5\\
            0&1&-2\\
            0&0&1\\
        \end{bmatrix}\\
    \end{split}
\end{equation*}

\subsubsection{rref}

~

\begin{equation*}
    \begin{split}
        &\begin{bmatrix}
            2&1&5\\
            0&1&-2\\
            0&0&1\\
        \end{bmatrix}\\
        \sim&\begin{bmatrix}
            1&\frac{1}{2}&\frac{5}{2}\\
            0&1&-2\\
            0&0&1\\
        \end{bmatrix}M_{1}(\frac{1}{2})\\
        \sim&\begin{bmatrix}
            1&0&\frac{7}{2}\\
            0&1&-2\\
            0&0&1\\
        \end{bmatrix}A_{21}(-\frac{1}{2})\\
        \sim&\begin{bmatrix}
            1&0&0\\
            0&1&-2\\
            0&0&1\\
        \end{bmatrix}A_{31}(-\frac{7}{2})\\
        \sim&\begin{bmatrix}
            1&0&0\\
            0&1&0\\
            0&0&1\\
        \end{bmatrix}A_{32}(2)\\
        &A\text{ can be reduced to }I\implies A\text{ is invertible}.\\
        \Rightarrow& A_{32}(2)A_{31}(-\frac{7}{2})A_{21}(-\frac{1}{2})M_{1}(\frac{1}{2})A=I\\
        \Rightarrow&A^{-1}=A_{32}(2)A_{31}(-\frac{7}{2})A_{21}(-\frac{1}{2})M_{1}(\frac{1}{2})\\
        \Rightarrow&A^{-1}=\begin{bmatrix}
            \frac{1}{2}&-\frac{1}{2}&-\frac{7}{2}\\
            0&1&2\\
            0&0&1\\
        \end{bmatrix}\\
    \end{split}
\end{equation*}

\subsubsection{Caley-Hamilton Theorem}

~

\begin{equation*}
    \begin{split}
        &p_A(\lambda)=\det(\lambda I-A)\\
        \Rightarrow&p_A(\lambda)=\lambda^3-4\lambda^2+5\lambda-2\\
        \Rightarrow&p_A(A)=A^3-4A^2+5A-2I=0\\
        \Rightarrow&A^{-1}=\frac{1}{2}(A^2-4A+5I)\\
        &A^{-1}=\begin{bmatrix}
            \frac{1}{2}&-\frac{1}{2}&-\frac{7}{2}\\
            0&1&2\\
            0&0&1\\
        \end{bmatrix}
    \end{split}
\end{equation*}

\newpage

\section{Problem 2}

\subsection{Question a}

~

\begin{equation*}
    \begin{split}
        &A^2:\\
        &(\lambda+2)(\lambda-3)=0\\
        \Rightarrow&\lambda^2-\lambda-6=0\\
        \Rightarrow&A^2-A-6I=0\\
        \Rightarrow&A^2=A+6I\\
        \Rightarrow&a_2=1,b_2=6\\
        A^n&=A^{n-1}A\\
        &=(a_{n-1}A+b_{n-1}I)A\\
        &=a_{n-1}A^2+b_{n-1}A\\
        &=a_{n-1}+6a_{n-1}I+b_{n-1}A\\
        &=(a_{n-1}+b_{n-1})A+6a_{n-1}I\\
        \Rightarrow&\begin{cases}
            a_n=a_{n-1}+b_{n-1}\\
            b_n=6a_{n-1}\\
        \end{cases}\\
        \Rightarrow&\begin{cases}
            a_n=a_{n-1}+6a_{n-2}\\
            b_n=6a_{n-1}\\
        \end{cases}\\
        &a_n=a_{n-1}+6a_{n-2}\\
        \Rightarrow&r^2-r-6=0\\
        \Rightarrow&\begin{cases}
            r_1=-2\\
            r_2=3\\
        \end{cases}\\
        \Rightarrow&a_n=(-2)^nA+3^nB\\
        &a_1=\frac{1}{6}b_2=1=-2A+3B\\
        &a_0=a_2-a_1=0=A+B\\
        \Rightarrow&\begin{cases}
            A=-\frac{1}{5}\\
            B=\frac{1}{5}\\
        \end{cases}\\
        \Rightarrow&\begin{cases}
            a_n=-\frac{1}{5}(-2)^n+\frac{1}{5}3^n\\
            b_n=6(-\frac{1}{5}(-2)^{n-1}+\frac{1}{5}3^{n-1})\\
        \end{cases}
    \end{split}
\end{equation*}

\subsection{Question b}

~

\begin{equation*}
    \begin{split}
        &A^{2023}=a_{2023}A+b_{2023}I\\
        &a_{2023}=-\frac{1}{5}(-2)^{2023}+\frac{1}{5}3^{2023}\\
        &b_{2023}=6(-\frac{1}{5}(-2)^{2022}+\frac{1}{5}3^{2022})\\
        \Rightarrow&A^{2023}=(-\frac{1}{5}(-2)^{2023}+\frac{1}{5}3^{2023})A+6(-\frac{1}{5}(-2)^{2022}+\frac{1}{5}3^{2022})I\\
    \end{split}
\end{equation*}

\newpage

\section{Problem 3}

\subsection{Question a}

~

\begin{equation*}
    \begin{split}
        &p_A(\lambda)=\det(\lambda I-A)\\
        \Rightarrow&p_A(\lambda)=a_0+a_1\lambda+...+a_n\lambda^n\\
        \Rightarrow&a_0 I+a_1A+...+a_nA^n=0\\
        \Rightarrow&A^n=-\frac{1}{a_n}(a_0I+a_1A+...+a_{n-1}A^{n-1})\\
        \Rightarrow&A^n\in \text{span}(I,A,A^2...,A^{n-1})\\
        \Rightarrow&\forall k>n: A^k=A^n\cdot A^{k-n}\in \text{span}(I,A,A^2...,A^{n-1})\\
        &\dim(\text{span}(I,A,A^2...,A^{n-1}))=n\Leftrightarrow I,A,A^2...,A^{n-1}\text{ are linear independent}\\
        \Rightarrow&\dim(W)\leq n
    \end{split}
\end{equation*}

\subsection{Question b}

~

The $\lambda$ in $p_A(\lambda)$ should be strictly be scalar, or the deterninant cannot be calculated.

\begin{equation*}
    \begin{split}
        &A\coloneqq\begin{bmatrix}
            1&2\\
            3&4\\
        \end{bmatrix}\\
        &p_A(\lambda)=\det(A-\lambda I)\\
        \Rightarrow&p_A(\lambda)=\det\begin{bmatrix}
            1-\lambda&2\\
            3&4-\lambda\\
        \end{bmatrix}\\
        &\text{Subtitute } \lambda \text{ as }A:\\
        &p_A(A)=\det\begin{bmatrix}
            1-\begin{bmatrix}
                1&2\\
                3&4\\
            \end{bmatrix}&2\\
            3&4-\begin{bmatrix}
                1&2\\
                3&4\\
            \end{bmatrix}\\
        \end{bmatrix}\\
        &\text{This is not a valid matrix, which makes the proof a wrong one.}
    \end{split}
\end{equation*}

\newpage

\section{Problem 4}

\subsection{Question a}

\subsubsection{Conjugate symmetry}

~

\begin{equation*}
    \begin{split}
        &\langle p(x),q(x)\rangle= p(-1)q(-1)+ p(1)q(1)+p(2)q(2)\\
        &\overline{\langle q(x),p(x)\rangle}=\overline{q(-1)p(-1)+q(1)p(1)+q(2)p(2)}\\
        &p(x)\land q(x)\in \R \implies \overline{q(-1)p(-1)+q(1)p(1)+q(2)p(2)}=q(-1)p(-1)+q(1)p(1)+q(2)p(2)\\
        \Rightarrow&\langle p(x),q(x)\rangle=\overline{\langle q(x),p(x)\rangle}\\
    \end{split}
\end{equation*}

\subsubsection{Linearity}

~

\begin{equation*}
    \begin{split}
        &\langle af(x)+bg(x),q(x)\rangle\\
        =&(af(-1)+bg(-1))q(-1)+(af(1)+bg(1))q(1)+(af(2)+bg(2))q(2)\\
        =&af(-1)q(-1)+af(1)q(1)+af(2)q(2)+bg(-1)q(-1)+bg(1)q(1)+bg(2)q(2)\\
        =&a(f(-1)q(-1)+f(1)q(1)+f(2)q(2))+b(g(-1)q(-1)+g(1)q(1)+g(2)q(2))\\
        =&a\langle f(x),q(x)\rangle+b\langle g(x),q(x)\rangle\\
    \end{split}
\end{equation*}

\subsubsection{Positive}

~

\begin{equation*}
    \begin{split}
        &\langle p(x),p(x)\rangle\\
        =&(p(-1))^2+(p(1))^2+(p(2))^2\geq0\\
    \end{split}
\end{equation*}

\subsubsection{Conclusion}

~

Since it satisfies all three axioms, it is a valid inner product.

\subsection{Question b}

\subsubsection{Positive}

~

\begin{equation*}
    \begin{split}
        &A\coloneqq\begin{bmatrix}
            a&b&c\\
            d&e&f\\
            g&h&i\\
        \end{bmatrix}\\
        &\langle A,A\rangle\\
        =&\text{Tr}(A^2)\\
        =&\text{Tr}(\begin{bmatrix}
            a&b&c\\
            d&e&f\\
            g&h&i\\
        \end{bmatrix}\begin{bmatrix}
            a&b&c\\
            d&e&f\\
            g&h&i\\
        \end{bmatrix})\\
        =&\text{Tr}(\begin{bmatrix}
            a^2+bd+cg&ad+be+ch&ac+bf+ci\\
            da+ed+fg&bd+e^2+fh&dc+ef+fi\\
            ga+hd+ig&gb+he+ih&cg+fh+i^2\\
        \end{bmatrix})\\
        =&a^2+bd+cg+bd+e^2+fh+cg+fh+i^2\\
        =&a^2+e^2+i^2+2(bd+cg+fh)\\
        \Rightarrow&\langle A,A\rangle\geq 0\Leftrightarrow a^2+e^2+i^2+2(bd+cg+fh)\geq0\\
        &\exists A\in M_{3\times3}(\R):a^2+e^2+i^2+2(bd+cg+fh)\leq0\\
        \Rightarrow&\langle A,A\rangle\text{ is not always }\geq 0\\
    \end{split}
\end{equation*}

\newpage

\section{Problem 5}

\subsection{Question a}

~

\begin{equation*}
    \begin{split}
        &\| u+v\|^2+\| u-v\|^2\\
        =&\langle u+v,u+v\rangle +\langle u-v,u-v\rangle\\
        =&\langle u,u+v\rangle+\langle v,u+v\rangle+\langle u,u-v\rangle-\langle v,u-v\rangle\\
        =&\overline{1}\langle u,u\rangle+\overline{1}\langle u,v\rangle+\overline{1}\langle v,u\rangle+\overline{1}\langle v,v\rangle+\overline{1}\langle u,u\rangle+\overline{-1}\langle u,v\rangle-\overline{1}\langle v,u\rangle-\overline{-1}\langle v,v\rangle\\
        =&2\langle u,u\rangle+2\langle v,v\rangle\\
        =&2\|u\|^2+2\|v\|^2\\
        =&2(\|u\|^2+\|v\|^2)
    \end{split}
\end{equation*}

\subsection{Question b}

\subsubsection{proof 1}

~

\begin{equation*}
    \begin{split}
        &\text{Cauchy-Schwatz Inequality}:\\
        &|\langle u,v\rangle|^2\leq\|u\|^2\|v\|^2\\
        \Rightarrow&\frac{|\langle u,v\rangle|^2}{\|u\|^2\|v\|^2}\leq1\\
        \Rightarrow&-1\leq\frac{|\langle u,v\rangle|}{\|u\|\|v\|}\leq1\\
    \end{split}
\end{equation*}

\subsubsection{proof 2}

~

\begin{equation*}
    \begin{split}
        &\|u-v\|^2\\
        =&\langle u-v,u-v\rangle\\
        =&\langle u,u-v\rangle-\langle v,u-v\rangle\\
        =&\overline{1}\langle u,u\rangle+\overline{-1}\langle u,v\rangle-\overline{1}\langle v,u\rangle-\overline{-1}\langle v,v\rangle\\
        =&\langle u,u\rangle+\langle v,v\rangle-2\langle u,v\rangle\\
        =&\|u\|^2+\|v\|^2-2\|u\|\|v\|\frac{|u,v|}{\|u\|\|v\|}\\
        =&\|u\|^2+\|v\|^2-2\|u\|\|v\|\cos \theta
    \end{split}
\end{equation*}

\subsection{Question c}

~

\begin{equation*}
    \begin{split}
        \|x^2+1\|=&\sqrt{\langle x^2+1,x^2+1\rangle}\\
        =&\sqrt{\int_{0}^{1}(x^2+1)^2dx}\\
        =&\sqrt{\frac{28}{15}}\\
        \|x-1\|=&\sqrt{\langle x-1,x-1\rangle}\\
        =&\sqrt{\int_{0}^{1}(x-1)^2dx}\\
        =&\sqrt{\frac{1}{3}}\\
        \|x+1\|=&\sqrt{\langle x+1,x+1\rangle}\\
        =&\sqrt{\int_{0}^{1}(x+1)^2dx}\\
        =&\sqrt{\frac{7}{3}}\\
    \end{split}
\end{equation*}

\begin{equation*}
    \begin{split}
        \langle x^2+1,x-1\rangle=&\int_{0}^{1}(x^2+1)(x-1)dx\\
        =&-\frac{7}{12}\\
        \langle x^2+1,x+1\rangle=&\int_{0}^{1}(x^2+1)(x+1)dx\\
        =&\frac{25}{12}\\
        \langle x-1,x+1\rangle=&\int_{0}^{1}(x-1)(x+1)dx\\
        =&-\frac{2}{3}\\
    \end{split}
\end{equation*}

\begin{equation*}
    \begin{split}
        \theta_{x^2+1,x-1}=&\arccos( \frac{\langle x^2+1,x-1\rangle}{\|x^2+1\|\|x-1\|})\\
        =&\arccos(\frac{-\frac{7}{12}}{\sqrt{\frac{28}{15}}\sqrt{\frac{1}{3}}})\\
        =&\arccos(-\frac{\sqrt{35}}{8})\\  
        \theta_{x^2+1,x+1}=&\arccos( \frac{\langle x^2+1,x+1\rangle}{\|x^2+1\|\|x+1\|})\\
        =&\arccos(\frac{\frac{25}{12}}{\sqrt{\frac{28}{15}}\sqrt{\frac{7}{3}}})\\
        =&\arccos(\frac{25\sqrt{5}}{56})\\
        \theta_{x-1,x+1}=&\arccos( \frac{\langle x-1,x+1\rangle}{\|x-1\|\|x+1\|})\\
        =&\arccos(\frac{-\frac{2}{3}}{\sqrt{\frac{1}{3}}\sqrt{\frac{7}{3}}})\\
        =&\arccos(-\frac{2}{\sqrt{7}})\\
    \end{split}
\end{equation*}

\newpage

\section{Reference}

~

Jeffery Shu

~

Frank Zhu


\end{document}